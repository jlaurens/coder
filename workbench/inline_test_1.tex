% !TEX program=lualatex
\documentclass{article}
\RequirePackage{inlinecode}
%\InlineCodeSet{minted}
\begin{document}
\section{No chunk}
\begin{itemize}
\item No name, lineno=1
\begin{InlineCode}{}
ABC
DEF
\end{InlineCode}
\item
lineno=3
\begin{InlineCode}{}
ABC
DEF
\end{InlineCode}
\item
lineno=1 with argument \texttt{reset}
\begin{InlineCode}{reset}
ABC
DEF
\end{InlineCode}
\end{itemize}
%\ExplSyntaxOn
%\tl_set:Nn \g_NLNCD_hook_tl {
%\noindent
%\NLNCD_if_show_name:TF { SHOW_NAME } { HIDE_NAME } \par\noindent
%\NLNCD_if_use_margin:TF { USE_MARGIN } { NOT_MARGIN } \par\noindent
%\NLNCD_if_only_top:TF { ONLY_TOP } { NOT_ONLY } \par\noindent
%}
%\g_NLNCD_hook_tl
%\ExplSyntaxOff
\section{Chunks}
\subsection{Only top}
\begin{itemize}
\item  Expected name A
\begin{InlineCode}{chunks=A}
ABC
DEF
\end{InlineCode}
\item Name A is hidden
\begin{InlineCode}{chunks=A}
ABC
DEF
\end{InlineCode}
\item Name A is back after \verb|\InlineCodeSet{only top=false}|
\InlineCodeSet{only top=false}
\begin{InlineCode}{chunks=A}
ABC
DEF
\end{InlineCode}
\item Name A is hidden after argument \verb|only top=true|
\begin{InlineCode}{chunks=A,only top=true}
ABC
DEF
\end{InlineCode}
\item Name A is still hidden after \verb|\InlineCodeSet{only top=true}|
\InlineCodeSet{only top=true}
\begin{InlineCode}{chunks=A}
ABC
DEF
\end{InlineCode}
\end{itemize}
\subsection{Name}
\InlineCodeSet{only top=false}
\begin{itemize}
\item Line numbering starts at 1
\begin{InlineCode}{chunks=B}
ABC
DEF
\end{InlineCode}
\item Line numbering continues
\begin{InlineCode}{chunks=B}
ABC
DEF
\end{InlineCode}
\end{itemize}
\subsection{No name}
\begin{itemize}
\item
\InlineCodeSet{name=false}
No name: \verb|\InlineCodeSet{name=false}|
\begin{InlineCode}{chunks=B}
ABC
DEF
\end{InlineCode}
\item
Expected B: Argument \verb|name=true|
\begin{InlineCode}{chunks=B, name=true}
ABC
DEF
\end{InlineCode}
\item
Expected B: \verb|\InlineCodeSet{name=true}|
\InlineCodeSet{name=true}
\begin{InlineCode}{chunks=B}
ABC
DEF
\end{InlineCode}
\item
No name: argument \verb|name=false|
\begin{InlineCode}{chunks=B,name=false}
ABC
DEF
\end{InlineCode}
\item
Back to default: expected B
\begin{InlineCode}{chunks=B}
ABC
DEF
\end{InlineCode}
\end{itemize}
\subsection{Multiple chunks}
\InlineCodeSet{name=true, only top=false}
\begin{itemize}
\item C, D
\begin{InlineCode}{chunks={C,D}}
ABC
DEF
\end{InlineCode}
C=D=3
\item C
\begin{InlineCode}{chunks=C}
ABC
DEF
\end{InlineCode}
C=5, D=3
\item D, C
\begin{InlineCode}{chunks={D, C}}
ABC
DEF
\end{InlineCode}
C=7, D=5
\item D
\begin{InlineCode}{chunks=D}
ABC
DEF
\end{InlineCode}
C=7, D=7
\item D, C
\begin{InlineCode}{chunks={D, C}}
ABC
DEF
\end{InlineCode}
C=9, D=9
\item C
\begin{InlineCode}{chunks={C}}
ABC
DEF
\end{InlineCode}
C=11, D=9
\item reset C
\begin{InlineCode}{chunks={C}, reset}
ABC
DEF
\end{InlineCode}
C=3, D=9
\end{itemize}
\section{Space}
BEFORE
\begin{InlineCode}{}
ABC
DEF
\end{InlineCode}
AFTER, THEN BIGGER
\begin{InlineCode}{sep=8pt plus 2pt minus 2pt}
ABC
DEF
\end{InlineCode}
BIGGER, THEN NORMAL
\begin{InlineCode}{}
ABC
DEF
\end{InlineCode}
BIGGER SKIP
\begin{InlineCode}{parskip=8pt plus 2pt minus 2pt}
ABC
DEF
\end{InlineCode}
NORMAL
\InlineCodeSet{}%
\begin{InlineCode}{}
ABC
DEF
\end{InlineCode}
\begin{InlineCode}{}
ABC
DEF
\end{InlineCode}
FIN
\section{Minted}
\end{document}


%\begin{InlineCode}{lang=python}
%f = 1
%for i in range(1,24):
%  f *= i
%print(f"The factorial of 23 is: {f}")
%\end{InlineCode}
%\noindent
%Without line numbers:
%\begin{InlineCode}{
%  lang=python,
%  no lineno,
%}
%f = 1
%for i in range(1,24):
%  f *= i
%print(f"The factorial of 23 is: {f}")
%\end{InlineCode}
%With chunk name:
%\begin{InlineCode}{
%  lang=python,
%  chunks=A,
%}
%f = 1
%for i in range(1,24):
%  f *= i
%print(f"The factorial of 23 is: {f}")
%\end{InlineCode}
%Without chunk name:
%\begin{InlineCode}{
%  lang=python,
%  chunks=A,
%  no name,
%}
%f = 1
%for i in range(1,24):
%  f *= i
%print(f"The factorial of 23 is: {f}")
%\end{InlineCode}
%Only top
%\InlineCodeSet{only top}
%\begin{InlineCode}{
%  chunks=B,
%}
%f = 1
%\end{InlineCode}
%\begin{InlineCode}{
%  chunks=B,
%}
%f = 1
%\end{InlineCode}

\end{document}
