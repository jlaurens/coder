% !TeX program=lualatex

\documentclass{article}

\usepackage[a4paper]{geometry}
\usepackage{fontspec}
\usepackage{xcolor}
\usepackage{hyperref}
\hypersetup{
    colorlinks,
    linkcolor={red!50!black},
    citecolor={blue!50!black},
    urlcolor={blue!80!black}
}
\usepackage{float}
\usepackage{tcolorbox}
\usepackage{fancyvrb-ex}
\usepackage{coder}
\usepackage{lipsum}
\usepackage{luacode} % JL
\usepackage{unicode-math}
\usepackage{multicol}
\usepackage{efbox}
\usepackage{mboxfill}

\typeout{************** REMOVE!!}
\CDRSet{python path=/Users/jlaurens/opt/anaconda3/bin/python}

\ExplSyntaxOn

\cs_new:Npn \cs #1 {
  \texttt{
    \textbackslash
    \tl_rescan:nn {
      \cctab_select:N \c_str_cctab
    } {
      #1
    }
  }
}
\cs_new:Npn \pkg { \textsf }

\cs_new:Npn \MyOption #1 {
  \texttt{#1}
}

\ExplSyntaxOff

\NewDocumentCommand \itemtt { o } {
  \IfNoValueTF { #1 } {
    \item
  } {
    \item[\ttfamily#1]
  }
}

\NewDocumentCommand \MyMeta {om} {%
  {\normalfont$⟨${\itshape#2}$⟩$%
  \IfValueT{#1}{\textsubscript{#1}}}%
}
\let\meta\MyMeta

\def\CDRCheckRed {}

\makeatletter
\def\CDR@Debug#1{\typeout{**** DEBUG #1}}
\makeatother
\begin{document}
\title{User manual for the \pkg{coder} package}
\author{Jérôme LAURENS\thanks{E-mail: jerome.laurens@u-bourgogne.fr}}
\maketitle
\section{Installation}
This package is part of any standard \LaTeX{} distribution. To use it in a document, put the next instruction in the preamble:
\begin{CDRBlock}[numbers=none,lang=latex,style=autumn]
\RequirePackage{coder}
\end{CDRBlock}
and run Lua\LaTeX{}. In order to have syntaxe highlighting like above, you must have \pkg{pygments} installed (see \url{https://pygments.org} for that purpose) but this is not a requirement for the other features of the \pkg{coder} package. To test \pkg{pygments} installation, you can run from a terminal
\\[0.25\baselineskip]
\verb|pygmentize -O full -o |\MyMeta{file name}\verb|-colorized.tex |\MyMeta{file name}\verb|.tex|
\\[0.25\baselineskip]
followed by
\\[0.25\baselineskip]
\verb|latex |\MyMeta{file name}\verb|-colorized.tex|.
\\[0.25\baselineskip]
The colorized code is then found in \MyMeta{file name}\verb|-colorized.pdf|%

To test the installation for an editor, the next document should output a small diagnostic page.
\CDRSet{numbers=none}
\begin{center}
\begin{tabular}{p{0.355\linewidth}|p{0.635\linewidth}}
\begin{minipage}[t]{0.9\linewidth}
\begin{CDRBlock}[lang=tex,tags=latex]
\documentclass{article}
\RequirePackage{coder}
\begin{document}
\CDRTest
\end{document}
\end{CDRBlock}
\end{minipage}%
\hspace{0.09\linewidth}
&
\begin{minipage}[t]{0.95\linewidth}
An example of a bad installation:
\par
Path to \textsf{python}: \texttt{/usr/bin/python}
\par
Path to \textsf{pygmentize}: \texttt{}
\par
Pygments is not available
\par\noindent
An example of a good installation:
\par
Path to \textsf{python}: \texttt{/opt/anaconda3/bin/python}
\par
Path to \textsf{pygmentize}: \texttt{/opt/anaconda3/bin/pygmentize}
\par\noindent
Pygments is available
\end{minipage}%
\end{tabular}
\end{center}

\section{Display code}
\subsection{Example}
To see \pkg{coder} in action, we compare \pkg{lua} and \pkg{python} syntax while computing recursively the factorial of a number.
\CDRSet{
  showspaces,
  only top,
  tags/lua={
    lang=lua,
    style=autumn,
    numbers=right,
    show tags=mirror,
  },
  tags/python={
    lang=python,
    style=trac,
    numbers=left,
    show tags=mirror,
  },
  tags/latex={
    lang=tex,
    style=default,
    numbers=left,
    show tags=none,
    escapeinside=⟨⟩,
  },
  tags/src={
    lang=tex,
    style=default,
    numbers=none,
    show tags=none,
    escapeinside=⟨⟩,
  }
}
\begin{figure}
\begin{center}
\begin{tabular}{p{0.495\linewidth}|p{0.495\linewidth}}
\begin{minipage}[t]{0.9\linewidth}
\begin{CDRBlock}[tags=lua]
function factorial(n)
  --[[Compute n!]]
  if n > 1 then
    return n * factorial(n-1)
  end
  return 1
end
\end{CDRBlock}
\end{minipage}%
\hspace{0.09\linewidth}
&
\hspace{0.09\linewidth}%
\begin{minipage}[t]{0.8\linewidth}
\begin{CDRBlock}[tags=python]
def factorial(n):
  '''Compute n!'''
  if n > 1:
    return n * factorial(n-1)
  return 1
\end{CDRBlock}
\end{minipage}%
\end{tabular}
\end{center}
\caption{First example}
\label{fig:First example}
\end{figure}
The coloring is made by \pkg{pygments} using its built in style \texttt{autumn} on the left and style \texttt{trac} on the right. Line numbering may appear on both sides of the code, or may be hidden.
%
\subsection{\cs{CDRCode} command}
Inserting code as inline text is possible with the command \CDRCode[tags=src]|\CDRCode| which syntax is
\begin{CDRBlock}[tags=src]
\CDRCode[⟨\MyMeta[1]{key}=\MyMeta[1]{value},\MyMeta[2]{key}=\MyMeta[2]{value},...⟩]?⟨\MyMeta{inline code chunk}⟩?
\end{CDRBlock}

Here the question mark \CDRCode|?| stands for quite any unicode character, at least one that is not in \MyMeta{inline code chunk} and is not a \CDRCode|[| of course.

The available options are collected in section \ref{sect:options} because they are shared.
\subsection{The \texttt{CDRBlock} environment}
Code chunks are put in a \texttt{CDRBlock} \LaTeX{} environment. Like many verbatim envitonments, the closing command is a full line literally consisting  of \CDRCode[lang=tex]|\end{CDRBlock}|, with no extra space.
Optional key--value options are enclosed with square brackets. The opening bracket must follow the \CDRCode[lang=tex]|\begin{CDRBlock}| instruction, on the very same line. For the first examples of figure \ref{fig:First example}, the input was
\begin{figure}[H]
\begin{center}
\begin{tabular}{p{0.495\linewidth}|p{0.495\linewidth}}
\begin{minipage}[t]{0.9\linewidth}
\begin{CDRBlock}[tags=src]
\begin{⟨CDRBlock⟩}[tags=lua]
function factorial(n)
  --[[Compute n!]]
  if n > 1 then
    return n * factorial(n-1)
  end
  return 1
end
\end{⟨CDRBlock⟩}
\end{CDRBlock}
\end{minipage}%
\hspace{0.09\linewidth}
&
\hspace{0.09\linewidth}%
\begin{minipage}[t]{0.8\linewidth}
\begin{CDRBlock}[tags=src]
\begin{⟨CDRBlock⟩}[tags=python]
def factorial(n):
  '''Compute n!'''
  if n > 1:
    return n * factorial(n-1)
  return 1
\end{⟨CDRBlock⟩}
\end{CDRBlock}
\end{minipage}%
\end{tabular}
\end{center}
\caption{Source of figure \ref{fig:First example}}
\end{figure}
%
\section{Decorations}
\subsection{Tags}
For the \CDRCode|CDRBlock| environment, the \CDRCode|show tags| option allows to display the list of tags.
The possible choices are illustrated hereafter below each source.
\setlength{\columnsep}{15mm}
\colorlet{CDROption}{magenta!66!black}
\begin{multicols}{3}
\CDRSet{
  tags/dummy = {
    frame=single,
    resetmargins=true,
  },
}
%
\begin{CDRBlock}[tags=src]
\begin{⟨CDRBlock⟩}[
  tags = dummy,
  ⟨\textcolor{CDROption}{\textbf{show tags=none}}⟩,
]
no tags
\end{⟨CDRBlock⟩}
\end{CDRBlock}
\begin{CDRBlock}[
  tags = dummy,
  show tags=none,
  frame=single,
]
no tags
\end{CDRBlock}
%
\begin{CDRBlock}[tags=src]
\begin{⟨CDRBlock⟩}[
  tags = dummy,
  ⟨\textcolor{CDROption}{\textbf{show tags=left}}⟩,
]
tags on the left
\end{⟨CDRBlock⟩}
\end{CDRBlock}
\begin{CDRBlock}[
  tags = dummy,
  show tags=left,
  frame=single,
]
tags on the left
\end{CDRBlock}
%
\begin{CDRBlock}[tags=src]
\begin{⟨CDRBlock⟩}[
  tags = dummy,
  ⟨\textcolor{CDROption}{\textbf{show tags=right}}⟩,
]
tags on the right
\end{⟨CDRBlock⟩}
\end{CDRBlock}
\begin{CDRBlock}[
  tags = dummy,
  show tags=right,
  frame=single,
]
tags on the right
\end{CDRBlock}
%
\begin{CDRBlock}[tags=src]
\begin{⟨CDRBlock⟩}[
  tags = dummy,
  ⟨\textcolor{CDROption}{\textbf{show tags=same}}⟩,
]
tags like numbers
\end{⟨CDRBlock⟩}
\end{CDRBlock}
\begin{CDRBlock}[
  tags = dummy,
  show tags=same,
  numbers=left,
  frame=single,
]
tags like numbers
\end{CDRBlock}
\begin{CDRBlock}[tags=src]
\begin{⟨CDRBlock⟩}[
  tags = dummy,
  ⟨\textcolor{CDROption}{\textbf{show tags=same}}⟩,
]
tags like numbers
\end{⟨CDRBlock⟩}
\end{CDRBlock}
\begin{CDRBlock}[
  tags = dummy,
  show tags=same,
  numbers=right,
  frame=single,
]
tags like numbers
\end{CDRBlock}
%
\begin{CDRBlock}[tags=src]
\begin{⟨CDRBlock⟩}[
  tags = dummy,
  ⟨\textcolor{CDROption}{\textbf{show tags=mirror}}⟩,
]
tags in mirror
\end{⟨CDRBlock⟩}
\end{CDRBlock}
\begin{CDRBlock}[
  tags = dummy,
  show tags=mirror,
  numbers=left,
  frame=single,
]
tags in mirror
\end{CDRBlock}
%
\begin{CDRBlock}[tags=src]
\begin{⟨CDRBlock⟩}[
  tags = dummy,
  ⟨\textcolor{CDROption}{\textbf{show tags=mirror}}⟩,
]
tags in mirror
\end{⟨CDRBlock⟩}
\end{CDRBlock}
\begin{CDRBlock}[
  tags = dummy,
  show tags=mirror,
  numbers=right,
  frame=single,
]
tags in mirror
\end{CDRBlock}
\end{multicols}
%
\subsection{Line numbering}
Line numbering only makes sense for blocks of code.
Options are illustrated below, to the right of each source.
\begin{center}
\setlength{\arraycolsep}{0pt}
\begin{tabular}{p{0.333\columnwidth}p{0.333\columnwidth}p{0.333\columnwidth}}
%
\begin{CDRBlock}[tags=src]
\begin{⟨CDRBlock⟩}[
  tags=X,
  show tags=none,
  numbers=left,
  ⟨\textcolor{CDROption}{\textbf{firstnumber=10}}⟩,
  ⟨\textcolor{CDROption}{\textbf{stepnumber=3}}⟩,
]
...
\end{⟨CDRBlock⟩}
\end{CDRBlock}
&
\begin{CDRBlock}[
  tags=X,
  show tags=none,
  numbers=left,
  firstnumber=10,
  stepnumber=3,
]
line 1 numbered 10
line 2
line 3 numbered 12
line 4
line 5
line 6 numbered 15
line 7
line 8
line 9 numbered 18
\end{CDRBlock}
&
\end{tabular}
\end{center}
Blocks with the same leftmost tag can be numbered continuously with option \CDRCode[tags=src]|firstnumber=last|, the next one starting just after the previous has stopped.
\begin{center}
\setlength{\arraycolsep}{0pt}
\begin{tabular}{p{0.333\columnwidth}p{0.333\columnwidth}p{0.333\columnwidth}}
%
\begin{CDRBlock}[tags=src]
\begin{⟨CDRBlock⟩}[
  tags=X,
  show tags=none,
  numbers=left,
  ⟨\textcolor{CDROption}{\textbf{firstnumber=last}}⟩,
  ⟨\textcolor{CDROption}{\textbf{stepnumber=4}}⟩,
]
...
\end{⟨CDRBlock⟩}
\end{CDRBlock}
&
\RenewDocumentCommand\CDRNumberMain{m}{%
  \color{CDROption}\bfseries#1
  \color{.!25!white}\makebox[4mm]{\mboxfill[0.75mm]{.}}%
}
\RenewDocumentCommand\CDRNumberOther{m}{%
  \color{lightgray}#1%
}
\begin{CDRBlock}[
  tags=X,
  show tags=none,
  numbers=left,
  firstnumber=last,
  stepnumber=4,
]
line 1
line 2 numbered 20
line 3
line 4
line 5
line 6 numbered 24
line 7
line 8
line 9
\end{CDRBlock}
&
\end{tabular}
\end{center}
%
Above, the intermediate line numbers were displayed by redefining the command \CDRCode[tags=src]|\CDRNumberOther| that defaults to no operation:
\begin{CDRBlock}[tags=src]
\RenewDocumentCommand\CDRNumberOther{m}{%
  \color{lightgray}#1%
}
\end{CDRBlock}
Similarly, the command \CDRCode[tags=src]|\CDRNumberMain| to display the main line numbers has been redefined with the help of the \pkg{mboxfill} package:
\begin{CDRBlock}[tags=src]
\RenewDocumentCommand\CDRNumberMain{m}{%
  \color{⟨\MyMeta{color}⟩}\bfseries#1
  \color{.!25!white}\makebox[4mm]{\mboxfill[0.75mm]{.}}%
}
\end{CDRBlock}
\subsection{Frames}
%
\subsubsection{Inline code}
When the \pkg{efbox} package is loaded, inline code can be framed by adding the option \CDRCode|engine=efbox|: \CDRCode[
  engine=efbox,
  efbox engine options={
    backgroundcolor=magenta!5!white,
    linecolor=magenta!80!black,
    linewidth=5\fboxrule,
  }
] |\textbf{ABCDE}|.
It is possible to pass any option of the \pkg{efbox} package with the key \CDRCode[mbox=false]|efbox engine options|, here is the source of the previous magenta box
\begin{CDRBlock}[tags=src]
\CDRCode[
  ⟨\textcolor{CDROption}{\textbf{engine}}⟩=efbox,
  ⟨\textcolor{CDROption}{\textbf{efbox engine options}}⟩={
    backgroundcolor=magenta!5!white,
    linecolor=magenta!80!black,
    linewidth=5\fboxrule,
  }
] |\textbf{ABCDE}|
\end{CDRBlock}
This engine named \CDRCode|efbox| was declared by next command
\begin{CDRBlock}[tags=src]
\CDRCodeEngineNew {⟨\MyMeta{engine name}⟩} {
  ⟨\MyMeta{engine instructions}⟩
}
\end{CDRBlock}
with \CDRCode|efbox| as \MyMeta{engine name},
and \CDRCode[tags=src]|\efbox[#1]{#2}| as \MyMeta{engine instructions}. It can be used to define a custom display engine as well.

There is an engine named \CDRCode[tags=src]|default| which is always available and used by default. \CDRCode[tags=src]|\CDRCodeEngineRenew| will be used to redefine engines, with a similar syntax.
%
\subsubsection{Block code}
If the \pkg{tcolorbox} package is loaded,
then blocks of code can be framed by adding the option \CDRCode|engine=tcbox|
\begin{CDRBlock}[
  label=Example,
  engine=tcbox,
  tcbox engine options={
    colback=magenta!5!white,
    colframe=magenta!80!black,
    boxrule=5\fboxrule,
    fonttitle=\sffamily\bfseries,
    title=\CDRGetOption{label},
  },
]
\textbf{ABCDE}
\end{CDRBlock}
It is possible to pass any option of the \pkg{tcolorbox} package with the key \CDRCode[mbox=false,tags=src]|tcbox engine options|, here is the source of the magenta box above
\begin{CDRBlock}[tags=src]
\begin{⟨CDRBlock⟩}[
  ⟨\textcolor{CDROption}{\textbf{label}}⟩=Example,
  ⟨\textcolor{CDROption}{\textbf{engine}}⟩=tcbox,
  ⟨\textcolor{CDROption}{\textbf{tcbox engine options}}⟩={
    colback=magenta!5!white,
    colframe=magenta!80!black,
    boxrule=5\fboxrule,
    fontupper=\sffamily\bfseries,
    title=\CDRGetOption{⟨\textcolor{CDROption}{\textbf{label}}⟩},
  },
]
\textbf{ABCDE}
\end{⟨CDRBlock⟩}
\end{CDRBlock}
This engine named \CDRCode|tcbox| was declared by next command
\begin{CDRBlock}[tags=src]
\CDRBlockEngineNew {⟨\MyMeta{engine name}⟩} {
  ⟨\MyMeta{begin engine instructions}⟩
} {
  ⟨\MyMeta{end engine instructions}⟩
}
\end{CDRBlock}
with \CDRCode|tcbox| as \MyMeta{engine name},
\CDRCode[tags=src]|\begin{tcolorbox}[#1]| as \MyMeta{begin engine instructions} and as \MyMeta{end engine instructions}
\CDRCode[tags=src]|\end{tcolorbox}|. This command is availale to define a custom display engine as well.

There is an engine named \CDRCode[tags=src]|default| which is always available and used by default. \CDRCode[tags=src]|\CDRBlockEngineRenew| will be used to redefine engines, with a similar syntax.
%
\begin{center}
\setlength{\arraycolsep}{0pt}
\begin{tabular}{p{0.495\columnwidth}p{0.495\columnwidth}}
The \pkg{coder} package also supports \pkg{fancyvrb} options to display frames, here is an example taken from the \pkg{fancyvrb} documentation:
\begin{CDRBlock}[
  frame=single,
  framerule=1mm,
  framesep=3mm,
  rulecolor=\color{red},
  fillcolor=\color{yellow}
]
Verbatim line.
\end{CDRBlock}
&
\begin{CDRBlock}[tags=src, no top space]
\begin{⟨CDRBlock⟩}[
  frame=single,
  framerule=1mm,
  framesep=3mm,
  rulecolor=\color{red},
  fillcolor=\color{yellow}
]
Verbatim line.
\end{⟨CDRBlock⟩}
\end{CDRBlock}
\end{tabular}
\end{center}

\section{Filtering}
\begin{center}
\setlength{\arraycolsep}{0pt}
\begin{tabular}{p{0.495\columnwidth}p{0.495\columnwidth}}
One can choose to display only some lines:
\begin{CDRBlock}[tags=src]
\begin{⟨CDRBlock⟩}[
  firstline=2,
  lastline=4,
]
Line 1
Line 2 *
Line 3 *
Line 4 *
Line 5
\end{⟨CDRBlock⟩}
\end{CDRBlock}
The output reads
\begin{CDRBlock}[
  firstline=2,
  lastline=-1,
]
Line 1
Line 2 *
Line 3 *
Line 4 *
Line 5
\end{CDRBlock}
&
Non positive integers would count from the last line: \CDRCode[tags=src]|lastline=-3| and \CDRCode[tags=src]|lastline=-1| would give the same output (notice that line numbering is 1 based).

Moreover, the option \CDRCode[tags=src]|firstline=2|
can be replaced by \CDRCode[tags=src]|firstline=L.*2| which means: from the first line that matches the regular expression pattern \CDRCode[tags=src]|"L.*2"| according to \LaTeX3 \url{interface3.pdf}.

Similarly, the option \CDRCode[tags=src]|lastline=4|
can be replaced by \CDRCode[tags=src]|lastline=L.*4| which means: up to the first line, from the line found above, that matches the regular expression pattern \CDRCode[tags=src]|"L.*2"|.
\end{tabular}
\end{center}
\end{document}      

%
\subsection{Options}
\label{sect:options}
\subsubsection{Shared by \cs{CDRCode} and \texttt{CDRBlock} environment}
\begin{description}
\itemtt[\CDRCheckRed tags={\{\meta[1]{tag},\meta[2]{tag},...\}}]
used for exportation, see section \ref{sect:exportation}, and also for styling, see section \ref{sect:styling}.
Initially a void list. 
\itemtt
\end{description}
%__tags, __engine, __pygments, default, 
\subsubsection{Specific to \cs{CDRCode}}
%    \g_CDR_tags_clist,
%    __code, default.code,
\begin{description}
\itemtt
\end{description}
\subsubsection{Specific to \texttt{CDRBlock} environment}
\begin{description}
\itemtt
\end{description}
%    __block, default.block, __pygments.block,
%    __fancyvrb.block __fancyvrb.frame, __fancyvrb.number, __fancyvrb,



\section{Export code with \cs{CDRExport}}
\label{sect:exportation}
\subsection{Usage}
The syntax for the \cs{CDRExport} command is the following, where the \CDRCode|file| key must be provided with a non empty value,
\begin{CDRBlock}[tags=src]
\CDRExport{
  file={⟨\MyMeta{output file path}⟩},
  tags={⟨{\color{CDROption}\MyMeta[1]{tag}},\MyMeta[2]{tag},...⟩},
}
\end{CDRBlock}
At the end of the typesetting process, all the code chunks with {\color{CDROption}\MyMeta[1]{tag}} like
\begin{CDRBlock}[tags=src]
\begin{⟨CDRBlock⟩}[tags={⟨\MyMeta{some tag},{\color{CDROption}\MyMeta[1]{tag}},\MyMeta{other tag},...⟩}]
...
\end{⟨CDRBlock⟩}
\end{CDRBlock}
are collected in the order of the source, then all the code chunks with \MyMeta[2]{tag} are collected in turn,... Finally, the result is written to \MyMeta{output file path}.

If the \CDRCode|tags| list is void, no exportation takes place. This exportation list can contain macros.
\subsection{Options}
\begin{description}
\itemtt[lang=\MyMeta{language}]
One of the languages \pkg{pygments} is aware of, the list of officially supported languages is available at \url{https://pygments.org/languages/}.
Initially \CDRCode|tex|.
Every subsequent code chunk which first tag is in the exportation list will have the same \CDRCode|lang| option, unil the next change. 
\itemtt[preamble=\MyMeta{preamble text}, preamble file=\MyMeta{preamble file path}]
The \MyMeta{preamble text} is saved before the collected code chunks unless the \CDRCode|raw| option is set to \CDRCode|true|. When a \MyMeta{preamble file path}
is given, then the \MyMeta{preamble text} is taken from the file instead.
Initially empty.
\itemtt[postamble=\MyMeta{postamble text}, postamble file=\MyMeta{postamble file path}]
the \MyMeta{postamble text} is saved after the code collected chunks unless the \CDRCode|raw| option is set to \CDRCode|true|. When a \MyMeta{preamble file path}
is given, then the \MyMeta{preamble text} is taken from the file instead. Initially empty.
\itemtt[escapeinside=\MyMeta{delimiters}]
the preamble and the postamble can contain \LaTeX\ macros enclosed between the delimiters,
these will be executed before exprtation.
Not any command is suitable here, macros containing text, \CDRCode|\date| may be convenient here.
When the \MyMeta{delimiters} is void, no escaping occurs.
When \MyMeta{delimiters} contains only one character, it is used both as opening and closing delimiter. When it contains at leat two characers, the first one is used as opening delimiter, the second one as closing delimiter.
\itemtt[raw{[=true$\vert$false]}]
When \CDRCode|true|, a preamble and postamble will be added to all the collected code chunks. Initially \CDRCode|false|.
\itemtt[once{[=true$\vert$false]}]
Consider for example the exportation \CDRCode|tags| list
\MyMeta[1]{tag},\MyMeta[2]{tag}.
If a block code chunk has exactly the same tag list, then it should be exported when \MyMeta[1]{tag} codes are collected but also when \MyMeta[2]{tag} codes are collected.
If the \CDRCode|once| option is set to \CDRCode|true|,
such a code chunk will be only be exported the first time, and will be ignored afterwards.
Initially \CDRCode|true|: export only once.
\end{description}
%    lang = tex,
%    preamble =,
%    postamble =,
%    raw = false,
%    once = true,

\end{document}
\section{Key--value options and tags}

Each time a code chunk is defined in a \CDRCode|CDRBlock| environment, it is recorded for further use.
Unless 
\subsection{About tags}
To each code chunck is associate an ordered list of tags, so far we have seen the tags ``lua'' and ``python'' appearing in the first examples opposite to the line numbers.
\section{Miscellanées}
\subsection{Cache}
In order to save typesetting time, the \pkg{coder} package asks \pkg{pygments} to colorize code only when necessary. The already colorized code is cached in the folder \verb|foo.pygd| for main source file \verb|foo.tex|.
With the option \verb|cache=false|, code colorization is computed on each run. If \pkg{pygments} is available, the cache folder is automatically emptied if there is no \verb|.aux| file.

\section{Cross references}
We can always refer to pages with the \MyOption{reflabel=\MyMeta{label name}}: 

\begin{center}
\begin{tabular}{p{0.495\linewidth}|p{0.495\linewidth}}
\begin{minipage}[t]{0.9\linewidth}
\begin{CDRBlock}[
  tags=latex,
  reflabel=My,
  escapeinside=||,
]
\begin{|CDRBlock|}[reflabel=My]
...
\end{|CDRBlock|}
\end{CDRBlock}
\end{minipage}%
\hspace{0.09\linewidth}
&
\hspace{0.09\linewidth}%
\begin{minipage}[t]{0.8\linewidth}
On the second typeset run, we have
\CDRCode[tags=latex,cache=false]|\pageref{My}| = \pageref{My}.
\end{minipage}%
\end{tabular}
\end{center}
Line references are only available when \pkg{pygments} is used: \MyOption{escapeinside} is used to define 2 characters that will delimit some \LaTeX{} instructions to be executed, here \CDRCode[tags=latex]|\label{line.421}|.
\begin{center}
\begin{tabular}{p{0.495\linewidth}|p{0.495\linewidth}}
\begin{minipage}[t]{0.9\linewidth}
\begin{CDRBlock}[tags=latex]
\begin{⟨CDRBlock⟩}[
  firstnumber=421,
  escapeinside=||,
]
line 421|\label{line.421}|
\end{⟨CDRBlock⟩}
\end{CDRBlock}
\end{minipage}%
\hspace{0.09\linewidth}
&
\hspace{0.09\linewidth}%
\begin{minipage}[t]{0.8\linewidth}
\begin{CDRBlock}[
  tags=latex,
  firstnumber=421,
  escapeinside=||,
]
line 421|\label{line.421}|
\end{CDRBlock}
On the second typeset run, we have
\CDRCode[tags=latex]|\ref*{line.421}| = \ref*{line.421}.
However this does not play well with \pkg{hyperref} hence the starred command.
\end{minipage}%
\end{tabular}
\end{center}

\end{document}

\title{}

\section{First examples}
%
%A simple verbatim text is the first example.
%
%\begin{Example}
%\begin{pygmented}{}
%Hello world!
%This is a simple demonstration text.
%\end{pygmented}
%\end{Example}
%
%The followig C program reads two integers and calculates their sum.
%
%\begin{Example}
%\begin{pygmented}{lang=c}
%#include <stdio.h>
%int main(void)
%{
%int a, b, c;
%printf("Enter two numbers to add: ");
%scanf("%d%d", &a, &b);
%c = a + b;
%printf("Sum of entered numbers = %d\n", c);
%return 0;
%}
%\end{pygmented}
%\end{Example}
%
%\begin{Example}
%In this program, \pyginline[lang=c]|int| is a type and
%\pyginline[lang=c]|"Enter two numbers to add: "| is a literal string.
%\end{Example}
%Next you can see a Java program to calculate the factorial of a number.
%
%\begin{Example}
%\inputpygmented[lang=java]{Factorial.java}
%\end{Example}
%
%\section{Choosing different Pygments styles}
%
%Instead of using the default style you may choose another stylesheet
%provided by Pygments by its name using the \verb|sty| option.
%
%To get a list of all available stylesheets, execute the following
%command on the command line:
%\begin{verbatim}
%$ pygmentize -L styles
%\end{verbatim}
%
%Creating your own styles is also very easy. Just follow the instructions
%provided on the website.
%
%As examples you can see a C program typeset with different styles.
%
%\begin{Example}
%\noindent
%\begin{minipage}[t]{0.49\linewidth}
%\begin{pygmented}{lang=c,gobble=4,sty=murphy}
%#include<stdio.h>
%main()
%{ int n;
%printf("Enter a number: ");
%scanf("%d",&n);
%if ( n%2 == 0 )
%printf("Even\n");
%else
%printf("Odd\n");
%return 0;
%}
%\end{pygmented}
%\end{minipage}
%\hfil
%\begin{minipage}[t]{0.49\linewidth}
%\begin{pygmented}{lang=c,gobble=4,sty=trac}
%#include<stdio.h>
%main()
%{ int n;
%printf("Enter a number: ");
%scanf("%d",&n);
%if ( n%2 == 0 )
%printf("Even\n");
%else
%printf("Odd\n");
%return 0;
%}
%\end{pygmented}
%\end{minipage}
%\end{Example}
%
%\section{Choosing a font}
%
%The value of the option \verb|font| is typeset before the content of the
%listing. Usualy it is used to specify a font to be used. See the
%following example.
%
%\begin{Example}
%\begin{pygmented}{lang=scala,font=\rmfamily\scshape\large}
%object bigint extends Application {
%def factorial(n: BigInt): BigInt =
%if (n == 0) 1 else n * factorial(n-1)
%
%val f50 = factorial(50); val f49 = factorial(49)
%println("50! = " + f50)
%println("49! = " + f49)
%println("50!/49! = " + (f50 / f49))
%}
%\end{pygmented}
%\end{Example}
%
%\begin{Example}
%\begin{pygmented}{lang=scala,font=\rmfamily\scshape\large}
%object bigint extends Application {
%def factorial(n: BigInt): BigInt =
%if (n == 0) 1 else n * factorial(n-1)
%
%val f50 = factorial(50); val f49 = factorial(49)
%println("50! = " + f50)
%println("49! = " + f49)
%println("50!/49! = " + (f50 / f49))
%}
%\end{pygmented}
%\end{Example}
%
%\section{Changing the background color}
%
%The option \verb|colback| can be used to choose a background color, as
%is shown in the folowing example.
%
%\begin{Example}
%\begin{pygmented}{lang=fsharp,colback=green!25}
%let rec factorial n =
%if n = 0
%then 1
%else n * factorial (n - 1)
%System.Console.WriteLine(factorial anInt)
%\end{pygmented}
%\end{Example}
%
%
%\section{Supressing initial characters}
%
%The option \verb|gobble| specifies the number of characters to suppress
%at the beginning of each line (up to a maximum of 9). This is mainly
%useful when environments are indented (Default: empty — no character
%suppressed).
%
%\begin{Example}
%A code snippet inside a minipage:
%\begin{minipage}[t]{.5\linewidth}
%\begin{pygmented}{lang=d,gobble=8}
%ulong fact(ulong n)
%{
%if(n < 2)
%return 1;
%else
%return n * fact(n - 1);
%}
%\end{pygmented}
%\end{minipage}
%\end{Example}
%
%
%\section{Size of tabulator}
%
%The option \verb|tabsize| specifies the number of of spaces given by a
%tab character (Default: 8).
%
%\begin{Verbatim}[showtabs,tabsize=1]
%\begin{pygmented}{lang=common-lisp,tabsize=4}
%;; Triple the value of a number
%(defun triple    (X)
%"Compute three times X."
%(* 3 X))
%\end{pygmented}
%\end{Verbatim}
%
%\begin{pygmented}{lang=common-lisp,tabsize=4}
%;; Triple the value of a number
%(defun triple    (X)
%"Compute three times X."
%(* 3 X))
%\end{pygmented}
%
%
%\section{Numbering lines}
%
%The lines of a listing can be numbered. The followig options control
%numbering of lines.
%\begin{itemize}
%\item Line numbering is enabled or disable with the \verb|linenos|
%boolean option.
%\item The number used for the first line can be set with the option
%\verb|linenostart|.
%\item The step between numbered lines can be set with the option
%\verb|linenostep|.
%\item The space between the line number and the line of the listing
%can be set with the option \verb|linenosep|.
%\end{itemize}
%
%In the followig listing you can see a Scheme function to calculate the
%factorial of a number.
%
%\begin{Example}
%\begin{pygmented}{lang=scheme,linenos,linenostart=1001,linenostep=2,linenosep=5mm}
%;; Building a list of squares from 0 to 9.
%;; Note: loop is simply an arbitrary symbol used as
%;; a label. Any symbol will do.
%
%(define (list-of-squares n)
%(let loop ((i n) (res '()))
%(if (< i 0)
%res
%(loop (- i 1) (cons (* i i) res)))))
%\end{pygmented}
%\end{Example}
%
%\section{Captioning}
%
%The option \verb|caption| can be used to set a caption for the listing.
%The option \verb|label| allows the assignment of a label to the listing.
%
%Here is an example:
%
%\begin{Example}
%\begin{pygmented}{lang=c++,label=lst:test,caption=A \textbf{C++} example}
%// This program adds two numbers and prints their sum.
%#include <iostream>
%int main()
%{
%int a;
%int b;
%int sum;
%sum = a + b;
%std::cout << "The sum of " << a << " and " << b
%<< " is " << sum << "\n";
%return 0;
%}
%\end{pygmented}
%\end{Example}
%
%\begin{Example}
%Listing \ref{lst:test} is a C++ program.
%\end{Example}
%
%\section{Escaping to \LaTeX{} inside a code snippet}
%
%The option \verb|texcomments|, if set to \texttt{true}, enables \LaTeX{}
%comment lines. That is, LaTex markup in comment tokens is not escaped
%so that \LaTeX{} can render it.
%
%The \verb|mathescape|, if set to \texttt{true}, enables \LaTeX{} math
%mode escape in comments. That is, \verb|$...$| inside a comment will
%trigger math mode.
%
%The option \verb|escapeinside|, if set to a string of length two,
%enables escaping to \LaTeX{}. Text delimited by these two characters
%is read as \LaTeX{} code and typeset accordingly. It has no effect in
%string literals. It has no effect in comments if \verb|texcomments| or
%\verb|mathescape| is set.
%
%Some examples follows.
%
%\begin{Example}
%\begin{pygmented}{lang=c++,texcomments}
%#include <iostream>
%using namespace std;
%main()
%{
%cout << "Hello World";  // prints \underline{Hello World}
%return 0;
%}
%\end{pygmented}
%\end{Example}
%
%\begin{Example}
%\begin{pygmented}{lang=python,mathescape}
%# Returns $\sum_{i=1}^{n}i$
%def sum_from_one_to(n):
%r = range(1, n + 1)
%return sum(r)
%\end{pygmented}
%\end{Example}
%
%\begin{Example}
%\begin{pygmented}{lang=c,escapeinside=||}
%
%if (|\textit{condition}|)
%|\textit{command$_1$}|
%else
%|\textit{command$_2$}|
%\end{pygmented}
%\end{Example}
%
%
%\section{Enclosing command and environment}
%
%After being prettified by Pygments, the listings are enclosed in a
%command (for \verb|\pyginline|) or in an environment (for
%\verb|pygmented| and \verb|includepygmented|). By default
%\verb|\pyginline| uses the command \verb|\efbox| from the \texttt{efbox}
%package, and \verb|pygmented| and \verb|includepygmented| use the
%environment \verb|mdframed| from the \texttt{mdframed} package.
%
%The enclosing command or environment should be configurable using a list
%of key-value pairs written between square brackets.
%
%The enclosing command for
%\verb|\pyginline| can be changed with the option
%\verb|inline method|. For instance, in the following the command
%\verb|\tcbox| from the \verb|tcolorbox| package is used:
%
%\begin{Example}
%In the previous Java program,
%\pyginline[lang=java,inline method=tcbox]|"Factorial of "| is a
%literal string.
%\end{Example}
%
%The enclosing environment for \verb|pygmented| and
%\verb|includepygmented| can be changed with the option
%\verb|boxing method|. For instance, here is a hello world program in
%C\#, enclosed in a \verb|tcolorbox| environment:
%
%\begin{Example}
%\begin{pygmented}{lang=csharp,boxing method=tcolorbox}
%using System;
%class Program
%{
%public static void Main(string[] args)
%{
%Console.WriteLine("Hello, world!");
%}
%}
%\end{pygmented}
%\end{Example}
%
%Any option unknown to Pygmen\TeX{} are passed to the enclosing command
%or environment.
%
%For instance:
%
%\begin{Example}
%\begin{pygmented}{lang=xml,boxing method=tcolorbox,colframe=red,boxrule=2mm}
%<!-- This is a note -->
%<note>
%<to>Tove</to>
%<from>Jani</from>
%<heading>Reminder</heading>
%<body>Don't forget me this weekend!</body>
%</note>
%\end{pygmented}
%\end{Example}
%
%\section{Setting global options for Pygmen\TeX{}}
%
%Global options can be setting using the \verb|setpygmented| command.
%See the examples that follows.
%
%\begin{Example}
%\setpygmented{lang=haskell, colback=red!30, font=\ttfamily\small}
%
%\begin{pygmented}{}
%sum :: Num a => [a] -> a
%sum [] = 0
%sum (x:xs) = x + sum xs
%\end{pygmented}
%\end{Example}
%
%\begin{Example}
%\begin{pygmented}{colback=blue!20, boxing method=tcolorbox}
%elem :: Eq a => a -> [a] -> Bool
%elem _ [] = False
%elem x (y:ys) = x == y || elem x ys
%\end{pygmented}
%\end{Example}
%
%\begin{Example}
%\setpygmented{lang=snobol}
%
%\begin{pygmented}{}
%OUTPUT = "What is your name?"
%Username = INPUT
%OUTPUT = "Thank you, " Username
%END
%\end{pygmented}
%\end{Example}
%
%\begin{Example}
%\setpygmented{test/.style={colback=yellow!33,boxing method=tcolorbox,colframe=blue}}
%
%\begin{pygmented}{test, lang=vbnet}
%Module Module1
%Sub Main()
%Console.WriteLine("Hello, world!")
%End Sub
%End Module
%\end{pygmented}
%\end{Example}
%
%\begin{Example}
%\begin{pygmented}{lang=tcl}
%puts "Hello, world!"
%\end{pygmented}
%\end{Example}
%
%\section{More examples of inline code snippets}
%
%\begin{Example}
%An inline source code snippet:
%\pyginline[lang=c]|const double alfa = 3.14159;|.
%This is a C declaration with initialization.
%\end{Example}

\typeout{DEBUG INLINE}

%\begin{Example}
\pyginline[lang=prolog,colback=yellow]=avo(A,B) :- pai(A,X), pai(X,B).=
is a Prolog clause. Its head is
\pyginline[lang=prolog,sty=emacs,colback=yellow,linecolor=red]=avo(A,B)=
and its body is
\pyginline[lang=prolog,sty=vim,colback=black,hidealllines]=pai(A,X), pai(X,B)=.
%\end{Example}
\end{document}


\begin{Example}
See the identifier \pyginline[inline method=efbox,colback=green!25]|variable|,
which names something. String literals in C looks like
\pyginline[lang=c,inline method=tcbox,colback=blue!20,boxrule=2pt]|"hello, world!\n"|.
\end{Example}

\setpygmented{colback=shadecolor}

\begin{Example}
This one
\pyginline[lang=ocaml,font=\ttfamily\scriptsize,topline=false] let x = [1;2;3] in length x:
is an OCaml expression with local bindings. With OCaml one can do
imperative, functional and object oriented programming.
\end{Example}

\begin{Example}
Now some Java code:
\pyginline[lang=java,sty=colorful,font=\ttfamily\itshape,linewidth=1pt]|public int f(double x)|.
This is a method header.
\end{Example}

\section{More examples of displayed code snippets}

\setpygmented{lang=scheme,colback=shadecolor,sty=emacs}

In listing \ref{lst:fact} you can see a function definition in the
Scheme language. This function computes the factorial of a natural
number.
\newline\rule{\linewidth}{2pt}
\begin{pygmented}{sty=emacs,
linenos,
label=lst:fact,
caption=A Scheme function.
}
(define fact
(lambda (n)
(if (= n 0)
1
(* n (fact (- n 1))))))
\end{pygmented}

Here you have some more code to further testing the package. Listing
\ref{lst:haskell} is a Haskell program. When run this program interacts
with the user asking the user name, reading a line input by the user,
and showing a greeting message to the user.

\inputpygmented[%
lang=haskell,
linenos,
linenostart=79831,
innerlinecolor=yellow, innerlinewidth=6pt,
middlelinecolor=blue, middlelinewidth=10pt,
outerlinecolor=green, outerlinewidth=12pt,
roundcorner=4,
colback=shadecolor,
caption=A haskell interactive program,
label=lst:haskell,
]{pygmentex_demo.hs}

This is a rule:

\noindent\rule{\linewidth}{2pt}

Now a Pascal procedure:

\inputpygmented[
lang=delphi,
linewidth=1.5pt,
font=\ttfamily\sffamily\large,
colback=yellow
]{pygmentex_demo.delphi}
and a Pascal program
\inputpygmented[lang=pascal,linenos,linenostart=5801]{pygmentex_demo.pas}

A Python code snippet:

\inputpygmented[
lang=python,
sty=emacs,
linenos,
linenostep=3,
linewidth=1pt,
colback=lightgreen
]{pygmentex_demo.py}

\section{Using code snippets in environments}

The following is a \textbf{description} environment.

\begin{description}
\item[An item] \lipsum[31]
\begin{pygmented}{lang=scala,colback=yellow,
% title=Item A
}
def qsort(xs: List[Int]): List[Int] =
xs match {
case Nil =>
Nil
case pivot :: tail =>
qsort(tail filter { _ < pivot }) :::
pivot :: qsort(tail filter { _ >= pivot })
}
\end{pygmented}
\lipsum[32]

\item[Another item] \lipsum[33]
\begin{pygmented}{lang=lua,colback=yellow}
function entry0 (o)
N=N + 1
local title = o.title or '(no title)'
fwrite('<LI><A HREF="#%d">%s</A>\n', N, title)
end
\end{pygmented}
\lipsum[34]
\end{description}

\section{A long program}

Here you can read the source code for a hand written lexical analyser
for the \emph{straight-line} programming language that I have developed
in Java.

\inputpygmented[boxing method=mdframed,lang=java,sty=autumn,colback=red!8,font=\ttfamily\small,tabsize=2,frametitle=\emph{Ad hoc} lexical analyser]{pygmentex_demo.java}

\section{Some fancy examples using \texttt{tcolorbox}}

The followig example uses \texttt{tcolorbox} to typeset the code
listing.

\newcounter{example}
\newlength{\examlen}
\colorlet{colexam}{red!75!black}

\begin{pygmented}{boxing method=tcolorbox,lang=scala,
title=Example \arabic{example}: hello from \texttt{Scala},
code={\refstepcounter{example}%
\settowidth{\examlen}{\Large\bfseries Example \arabic{example}}},%
coltitle=colexam,fonttitle=\Large\bfseries,
enhanced,breakable,
%before=\par\medskip,
parbox=false,
frame hidden,interior hidden,segmentation hidden,
boxsep=0pt,left=0pt,right=3mm,toptitle=2mm,pad at break=0mm,
overlay unbroken={\draw[colexam,line width=1pt] (frame.north west)
--([xshift=-0.5pt]frame.north east)--([xshift=-0.5pt]frame.south east)
--(frame.south west);
\draw[colexam,line width=2pt] ([yshift=0.5pt]frame.north west)
-- +(\examlen,0pt);},
overlay first={\draw[colexam,line width=1pt] (frame.north west)
--([xshift=-0.5pt]frame.north east)--([xshift=-0.5pt]frame.south east);
\draw[red!75!black,line width=2pt] ([yshift=0.5pt]frame.north west)
-- +(\examlen,0pt);},
overlay middle={\draw[colexam,line width=1pt] ([xshift=-0.5pt]frame.north east)
--([xshift=-0.5pt]frame.south east); },
overlay last={\draw[colexam,line width=1pt] ([xshift=-0.5pt]frame.north east)
--([xshift=-0.5pt]frame.south east)--(frame.south west);}%
}
object HelloWorld extends App {
println("Hello, world!")
}\end{pygmented}

\begin{pygmented}{boxing method=tcolorbox,lang=java,
enhanced,colback=blue!10!white,colframe=orange,top=4mm,
enlarge top by=\baselineskip/2+1mm,
enlarge top at break by=0mm,pad at break=2mm,
fontupper=\normalsize,
overlay unbroken and first={%
\node[rectangle,rounded corners,draw=black,fill=blue!20!white,
inner sep=1mm,anchor=west,font=\small]
at ([xshift=4.5mm]frame.north west) {\strut\textbf{My fancy title}};},
}
public class Hello {
public static void main(String[] args) {
System.out.println("Hello, world!")
}
}
\end{pygmented}

\begin{pygmented}{boxing method=tcolorbox,lang=haskell,
enhanced,sharp corners=uphill,
colback=blue!25!white,colframe=blue!25!black,coltext=blue!90!black,
fontupper=\Large\bfseries,arc=6mm,boxrule=2mm,boxsep=5mm,
borderline={0.3mm}{0.3mm}{white}
}
module Main (main) where

main :: IO ()
main = putStrLn "Hello, world!"
\end{pygmented}

\begin{pygmented}{boxing method=tcolorbox,lang=c++,
enhanced,frame style image=blueshade.png,
opacityback=0.75,opacitybacktitle=0.25,
colback=blue!5!white,colframe=blue!75!black,
title=My title
}
#include <iostream>
using namespace std;
int main(int argc, char** argv) {
cout << "Hello, world!" << endl;
return 0;
}
\end{pygmented}

\begin{pygmented}{boxing method=tcolorbox,lang=d,
enhanced,attach boxed title to top center={yshift=-3mm,yshifttext=-1mm},
colback=blue!5!white,colframe=blue!75!black,colbacktitle=red!80!black,
title=My title,fonttitle=\bfseries,
boxed title style={size=small,colframe=red!50!black}
}
/* This program prints a
hello world message
to the console.  */

import std.stdio;

void main()
{
writeln("Hello, World!");
}
\end{pygmented}


\section{Some fancy examples using \texttt{mdframed}}

The followig example uses \texttt{mdframed} to typeset the code listing.

\global\mdfdefinestyle{exampledefault}{%
linecolor=red,linewidth=3pt,%
leftmargin=1cm,rightmargin=1cm
}

\begin{pygmented}{boxing method=mdframed,lang=ada,style=exampledefault}
with Ada.Text_IO;

procedure Hello_World is
use Ada.Text_IO;
begin
Put_Line("Hello, world!");
end;
\end{pygmented}

\global\mdfapptodefinestyle{exampledefault}{%
topline=false,bottomline=false,
}

\begin{pygmented}{boxing method=mdframed,lang=pascal,style=exampledefault,frametitle={Saying \emph{hello} from Pascal}}
program HelloWorld;

begin
WriteLn('Hello, world!');
end.
\end{pygmented}

\global\mdfdefinestyle{separateheader}{%
frametitle={%
\tikz[baseline=(current bounding box.east),outer sep=0pt]
\node[anchor=east,rectangle,fill=blue!20]
{\strut Saying \emph{hello} in Modula-2};},
innertopmargin=10pt,linecolor=blue!20,%
linewidth=2pt,topline=true,
frametitleaboveskip=\dimexpr-\ht\strutbox\relax,
frametitlerule=false,
backgroundcolor=white,
}

\begin{pygmented}{boxing method=mdframed,lang=modula2,style=separateheader}
MODULE Hello;
FROM STextIO IMPORT WriteString;
BEGIN
WriteString("Hello World!");
END Hello.
\end{pygmented}


\tikzset{titregris/.style =
{draw=gray, thick, fill=white, shading = exersicetitle, %
text=gray, rectangle, rounded corners, right,minimum height=.7cm}}
\pgfdeclarehorizontalshading{exersicebackground}{100bp}
{color(0bp)=(green!40); color(100bp)=(black!5)}
\pgfdeclarehorizontalshading{exersicetitle}{100bp}
{color(0bp)=(red!40);color(100bp)=(black!5)}
\newcounter{exercise}
\renewcommand*\theexercise{Exercise~n\arabic{exercise}}
\makeatletter
\def\mdf@@exercisepoints{}%new mdframed key:
\define@key{mdf}{exercisepoints}{%
\def\mdf@@exercisepoints{#1}
}
\mdfdefinestyle{exercisestyle}{%
outerlinewidth=1em,outerlinecolor=white,%
leftmargin=-1em,rightmargin=-1em,%
middlelinewidth=1.2pt,roundcorner=5pt,linecolor=gray,
apptotikzsetting={\tikzset{mdfbackground/.append style ={%
shading = exersicebackground}}},
innertopmargin=1.2\baselineskip,
skipabove={\dimexpr0.5\baselineskip+\topskip\relax},
skipbelow={-1em},
needspace=3\baselineskip,
frametitlefont=\sffamily\bfseries,
settings={\global\stepcounter{exercise}},
singleextra={%
\node[titregris,xshift=1cm] at (P-|O) %
{~\mdf@frametitlefont{\theexercise}\hbox{~}};
\ifdefempty{\mdf@@exercisepoints}%
{}%
{\node[titregris,left,xshift=-1cm] at (P)%
{~\mdf@frametitlefont{\mdf@@exercisepoints points}\hbox{~}};}%
},
firstextra={%
\node[titregris,xshift=1cm] at (P-|O) %
{~\mdf@frametitlefont{\theexercise}\hbox{~}};
\ifdefempty{\mdf@@exercisepoints}%
{}%
{\node[titregris,left,xshift=-1cm] at (P)%
{~\mdf@frametitlefont{\mdf@@exercisepoints points}\hbox{~}};}%
},
}
\makeatother

\begin{pygmented}{boxing method=mdframed,lang=go,style=exercisestyle}
// hello world in 'go'
package main

import "fmt"

func main() {
fmt.Println("Hello, world!")
}
\end{pygmented}

\begin{pygmented}{boxing method=mdframed,lang=objective-c,style=exercisestyle,exercisepoints=10}
/* hello from objective-c */

#import <stdio.h>
#import <Foundation/Foundation.h>

int main(void)
{
NSLog(@"Hello, world!\n");
return 0;
}
\end{pygmented}

\mdfdefinestyle{another}{%
linecolor=red,middlelinewidth=2pt,%
frametitlerule=true,%
apptotikzsetting={\tikzset{mdfframetitlebackground/.append style={%
shade,left color=white, right color=blue!20}}},
frametitlerulecolor=green!60,
frametitlerulewidth=1pt,
innertopmargin=\topskip,
}

\begin{pygmented}{boxing method=mdframed,lang=c,style=another,frametitle={Hello from C}}
#include <stdio.h>
int main(int argc, char **argv) {
printf("Hello, world!\n");
return 0;
}
\end{pygmented}


\section{Conclusion}

That is all.

\end{document}

