\documentclass{article}
\directlua{
  CDR = {}
debug_msg = function(...)
  print('**** CDR ****', ...)
end
}
\newtoks\X
\newtoks\Y
\begin{document}
\section{preparation}
\directlua{
  local f = function(s1, s2)
    tex.print(s1);
    tex.print(s2);
    tex.toks.X = s1;
    tex.print([[\string\the\string\X]]);    
    tex.toks.X = s2;
    tex.print([[\string\the\string\X]]);    
  end
  f('ABCDE', 'EDCBA');
}
We get a EDCBA too many times.
\section{coroutine skeletton}
\directlua{
  co = coroutine.create(function(s1, s2)
    tex.toks.X = s1;
    tex.print([[\string\the\string\X]]);
    coroutine.yield()
    tex.toks.X = s2;
    tex.print([[\string\the\string\X]]);
  end);
  coroutine.resume(co, 'ABCDE', 'EDCBA');
}
We got ABCDE once.
\directlua{
  coroutine.resume(co);
}
We got EDCBA once.
\section{coroutine skeletton 2}
\directlua{
  co = coroutine.create(function(s1, s2)
    tex.toks.X = s1;
    tex.print([[\string\the\string\X]]);
    coroutine.yield()
    tex.toks.X = s2;
    tex.print([[\string\the\string\X]]);
  end);
  coroutine.resume(co, 'ABCDE', 'EDCBA');
  coroutine.resume(co);
  coroutine.resume(co);
  coroutine.resume(co);
}
We got EDCBA twice; no matter how many times the coroutine is resumed, as long as it is more than 2.

\section{coroutine skeletton 3}
\directlua{
  co = coroutine.create(function(s1, s2)
    tex.toks.X = s1;
    tex.print([[\string\the\string\X]]);
    coroutine.yield()
    tex.toks.X = s2;
    tex.print([[\string\the\string\X]]);
  end);
  coroutine.resume(co, 'ABCDE', 'EDCBA');
  tex.print([[\string\directlua{coroutine.resume(co)}]]);
}
We got ABCDE and EDCBA.
\section{coroutine skeletton 4}
One coroutine, one \string\directlua\\
\directlua{
  co = coroutine.create(function(s1, s2)
    tex.toks.X = s1;
    tex.print([[\string\the\string\X]]);
    tex.print([[\string\directlua{coroutine.resume(co)}]]);
    coroutine.yield()
    tex.toks.X = s2;
    tex.print([[\string\the\string\X]]);
  end);
  coroutine.resume(co, 'ABCDE', 'EDCBA');
}
We got ABCDE and EDCBA.
\section{coroutine skeletton 4bis}
Two coroutines, one \string\directlua\\
\directlua{
  cox = coroutine.create(function(s1, s2)
    tex.toks.X = s1;
    texio.write_nl('body 1: start ', tex.toks.X);
    tex.print([[\string\the\string\X]]);
    tex.print([[\string\directlua{
      texio.write_nl('body 1: before resume ', tex.toks.X);
      coroutine.resume(cox);
      texio.write_nl('body 1: after resume ', tex.toks.X);
    }]]);
    texio.write_nl('body 1: before yield ', tex.toks.X);
    coroutine.yield();
    texio.write_nl('body 1: after yield ', tex.toks.X);
    tex.toks.X = s2;
    tex.print([[\string\the\string\X]]);
    texio.write_nl('body 1: done');
    texio.write_nl('');
  end);
  coroutine.resume(cox, 'ABCDE', 'EDCBA');
  coy = coroutine.create(function(s1, s2)
    tex.toks.Y = s1;
    texio.write_nl('body 2: start '..tex.toks.Y)
    tex.print([[\string\the\string\Y]]);
    tex.print([[\string\directlua{
      texio.write_nl('body 2: before resume ', tex.toks.Y);
      coroutine.resume(coy);
      texio.write_nl('body 2: after resume ', tex.toks.Y);
      texio.write_nl('');
    }]]);
    texio.write_nl('body 2: before yield '..tex.toks.Y);
    coroutine.yield();
    texio.write_nl('body 2: after yield '..tex.toks.Y);
    tex.toks.Y = s2;
    tex.print([[\string\the\string\Y]]);
    texio.write_nl('body 2: done '..tex.toks.Y);
  end);
  coroutine.resume(coy, '12345', '54321');
  texio.write_nl('');
}
We got ABCDE and EDCBA.
\section{coroutine skeletton 5}
\directlua{
  coroutines = {}
  n1 = \string#coroutines+1
  coroutines[n1] = coroutine.create(function(s1, s2)
    tex.toks.X = s1;
    tex.print([[\string\the\string\X]]);
    tex.print([[\string\directlua{coroutine.resume(coroutines[n1])}]]);
    coroutine.yield()
    tex.toks.X = s2;
    tex.print([[\string\the\string\X]]);
  end);
  coroutine.resume(coroutines[n1], 'ABCDE', 'EDCBA');
  n2 = \string#coroutines+1;
  coroutines[n2] = coroutine.create(function(s1, s2)
    tex.toks.X = s1;
    tex.print([[\string\the\string\X]]);
    tex.print([[\string\directlua{coroutine.resume(coroutines[n2])}]]);
    coroutine.yield()
    tex.toks.X = s2;
    tex.print([[\string\the\string\X]]);
  end);
  coroutine.resume(coroutines[n2], '12345', '54321');
}
We got ABCDE and EDCBA.
\section{coroutine skeletton 6}
\directlua{
  helper = {};
  do
    coroutines = {};
    helper = function (f)
      local n = 1+\string#coroutines;
      coroutines[n] = coroutine.create(f);
      return function(...)
        tex.print('START', n, ...)
        coroutine.resume(coroutines[n], ...);
      end, function ()
        tex.print('RESUME', n)
        tex.print([[\string\directlua]]..'{coroutine.resume(coroutines['..n..'])}');
        coroutine.yield()
      end
    end
  end
  local resume, synchronize;
  resume, synchronize = helper(function(s1, s2)
    tex.toks.X = s1;
    tex.print([[\string\the\string\X]]);
    synchronize();
    tex.toks.X = s2;
    tex.print([[\string\the\string\X]]);
  end);
  resume('ABCDE', 'EDCBA');
  tex.print('\string\\\string\\ SECOND TEST\string\\\string\\')
  local resume6, synchronize6;
  resume6, synchronize6 = helper(function(s1, s2)
    tex.toks.Y = s1;
    tex.print([[\string\the\string\Y]]);
    synchronize6();
    tex.toks.Y = s2;
    tex.print([[\string\the\string\Y]]);
  end);
  resume6('12345', '54321');
}

We were expected ABCDE and EDCBA.
We were expected 12345 and 54321.
%
\section{coroutine skeletton 7}
\directlua{
  local resume, synchronize;
  resume, synchronize = helper(function(s1, s2)
    tex.toks.X = s1;
    tex.print([[\string\the\string\X]]);
    synchronize();
    tex.toks.X = s2;
    tex.print([[\string\the\string\X]]);
  end);
  resume('ABCDE', 'EDCBA');
}
We got ABCDE, EDCBA.
\section{coroutine skeletton 8}
\directlua{
  CDR = {}
  do
    coroutines = {};
    function CDR:coroutine_create (f)
      local n = 1+\string#coroutines;
      coroutines[n] = coroutine.create(f);
      return function(...)
        coroutine.resume(coroutines[n], ...);
      end, function ()
        tex.print([[\string\directlua]]..'{coroutine.resume(coroutines['..n..'])}');
        coroutine.yield()
      end
    end
  end
  local resume, synchronize;
  resume, synchronize = CDR:coroutine_create(function(s1, s2)
    tex.toks.X = s1;
    tex.print([[\string\the\string\X]]);
    synchronize();
    tex.toks.X = s2;
    tex.print([[\string\the\string\X]]);
  end);
  resume('ABCDE', 'EDCBA');
  tex.print('\string\\\string\\ SECOND TEST\string\\\string\\')
  local resume6, synchronize6;
  resume6, synchronize6 = CDR:coroutine_create(function(s1, s2)
    tex.toks.Y = s1;
    tex.print([[\string\the\string\Y]]);
    synchronize6();
    tex.toks.Y = s2;
    tex.print([[\string\the\string\Y]]);
  end);
  resume6('12345', '54321');
}

We were expected ABCDE and EDCBA.
We were expected 12345 and 54321.
%
\section{coroutine skeletton 8 bis}
\directlua{
  CDR = {}
  do
    Cs = {};
    function CDR:coroutine_resume(n)
      coroutine.resume(Cs[n].co)
    end;
    function CDR:coroutine_create (f)
      local n = 1+\string#Cs;
      Cs[n] = {
        co = coroutine.create(f)
      };
      return {
        resume = function(this, ...)
          coroutine.resume(Cs[n].co, ...);
        end,
        synchronize = function (this)
          tex.print([[\string\directlua]]..'{CDR:coroutine_resume('..n..')}');
          coroutine.yield()
        end,
      }
    end
  end
  local  C8;
  C8 = CDR:coroutine_create(function(s1, s2)
    tex.toks.X = s1;
    tex.print([[\string\the\string\X]]);
    C8:synchronize();
    tex.toks.X = s2;
    tex.print([[\string\the\string\X]]);
  end);
  C8:resume('ABCDE', 'EDCBA');
}

We were expected ABCDE and EDCBA.
%
\section{coroutine skeletton 8 ter}
\directlua{
  CDR = {}
  do
    Cs = {};
    function CDR:coroutine_resume(n)
      coroutine.resume(Cs[n].co)
    end;
    function CDR:coroutine_create (f)
      local n = 1+\string#Cs;
      local ans = {
        co = coroutine.create(f),
        resume = function(this, ...)
          coroutine.resume(Cs[n].co, ...);
        end,
        synchronize = function (this)
          tex.print([[\string\directlua]]..'{CDR:coroutine_resume('..n..')}');
          coroutine.yield()
        end,
      };
      Cs[n] = ans;
      return ans
    end
  end
  local  C8;
  C8 = CDR:coroutine_create(function(s1, s2)
    tex.toks.X = s1;
    tex.print([[\string\the\string\X]]);
    C8:synchronize();
    tex.toks.X = s2;
    tex.print([[\string\the\string\X]]);
  end);
  C8:resume('ABCDE', 'EDCBA');
}

We were expected ABCDE and EDCBA.
%
\section{coroutine skeletton 8 quattro}
\directlua{
  CDR = {}
  do
    Cs = {};
    function CDR:thread_resume(n)
      coroutine.resume(Cs[n].co)
    end;
    function CDR:coroutine_create (f)
      local n = 1+\string#Cs;
      Cs[n] = {
        co = coroutine.create(function (...)
          local ans = { f(...) }
          Cs[n] = nil;
          return table.unpack(ans)
        end),
        fire = function(this, ...)
          if not already then
            already = true;
            coroutine.resume(Cs[n].co, ...);
          else
            texio.write_nl('CDR: Already fired')
          end;
        end,
        synchronize = function (this)
          tex.print([[\string\directlua]]..'{CDR:thread_resume('..n..')}');
          coroutine.yield()
        end,
      };
      return Cs[n]
    end
  end
  local  C8;
  C8 = CDR:coroutine_create(function(s1, s2)
    tex.toks.X = s1;
    tex.print([[\string\the\string\X]]);
    C8:synchronize();
    tex.toks.X = s2;
    tex.print([[\string\the\string\X]]);
  end);
  C8:fire('ABCDE', 'EDCBA');
  C8:fire('ABCDE', 'EDCBA');
}

We were expected ABCDE and EDCBA.
%
%
\section{OLD coroutine}
\directlua{
  CDR = {}
debug_msg = function(...)
  texio.write_nl('');
  print('**** CDR ****', ...);
end
do
  local coroutines = {}
  function CDR:coroutine_create(f)
    local n = \string#coroutines + 1;
debug_msg('CDR:coroutine_... CREATE', n);
    coroutines[n] = coroutine.create(function (...)
debug_msg('CDR:coroutine_... START', n);
      local ans = { f(...) };
debug_msg('CDR:coroutine_... REMOVE', n);
      coroutines[n] = nil;
debug_msg('CDR:coroutine_... RETURN', n)
      return table.unpack(ans);
    end);
    return function (...)
debug_msg('CDR:coroutine_... RESUME', n);
      return coroutine.resume(coroutines[n], ...);
    end, function (...)
      tex.sprint([[\string\directlua]]..'{coroutine.resume(coroutines['..n..'])}')
debug_msg('CDR:coroutine_... YIELD', n)
      return coroutine.yield(...)
    end
  end;
end
}
POC

\directlua{
  texio.write_nl('**** coroutine');
  local resume10, synchronize10;
  resume10, synchronize10 = CDR:coroutine_create(function()
    tex.print('ABCDE');
    synchronize10();
    tex.print('EDCBA');
    tex.print('DONE');
  end);
  resume10();
}

POC

\section{Lua example}
\directlua{
  function foo (a)
       print("foo", a);
       return coroutine.yield(2*a)
     end
     
     co = coroutine.create(function (a,b)
           print("co-body", a, b)
           local r = foo(a+1)
           print("co-body", r)
           local r, s = coroutine.yield(a+b, a-b)
           print("co-body", r, s)
           return b, "end"
     end)
     
     print("main", coroutine.resume(co, 1, 10));
     print("main", coroutine.resume(co, "r"));
     print("main", coroutine.resume(co, "x", "y"));
     print("main", coroutine.resume(co, "x", "y"));
}

\end{document}
