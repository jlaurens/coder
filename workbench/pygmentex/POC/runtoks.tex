\documentclass{article}
\begin{document}
\section{\string\newtoks}
The <token register> is the number or the name.
\newtoks\MyToks
\directlua{
  myToks=\the\count15;
  tex.print(myToks);
}
\MyToks{ABC}
\directlua{
  tex.runtoks(myToks);
  tex.settoks(myToks,'CBA');
}
\the\MyToks
\MyToks{ABCD}
\directlua{
  tex.runtoks('MyToks');
  tex.settoks('MyToks','DCBA');
}
\the\MyToks
\MyToks{ABCDE}
\directlua{
  tex.print(tex.toks.MyToks);
  tex.toks.MyToks='EDCBA';
  tex.runtoks(myToks);
  tex.runtoks(myToks);
}
\the\MyToks
\section{\string\scantoks}
\the\MyToks
\directlua{
  tex.print(tex.toks.MyToks);
  tex.scantoks(0,myToks,"e=mc");
  tex.runtoks(myToks);
  tex.runtoks(0);
  tex.scantoks('MyToks',0,"$e=mc^2$");
}
\the\MyToks
\directlua{
  tex.toks.MyToks=[[\string\textbf{ABCDE}]];
  tex.runtoks(myToks);
  tex.scantoks('MyToks',0,[[\string\textbf{ABCDE}]]);
  tex.runtoks(myToks);
}
\section {\string\input}
runtoks is restricted.
The \string\input\ may works only in certain circumstances that I did not find.
\directlua{
  luatexbase.add_to_callback(
    'open_read_file',
    function (file_name)
      texio.write_nl('');
      texio.write_nl('**** OPEN READ FILE '..file_name);
      local i = 0
      return {
        reader = function ()
          i = i+1;
          texio.write_nl('READER LINE '..i);
          if i < 5 then
            return 'READER LINE '..i
          end
        end
      }
    end,
    'CDR runtoks input'
  );
}
RAW LINE 1
RAW LINE 2
RAW LINE 3

Here is the test:
\directlua{
  texio.write_nl('From scantoks');
  tex.scantoks('MyToks',0,[[\string\textbf{BEFORE}\stringRAW LINE 1
RAW LINE 2
RAW LINE 3
\string\textbf{AFTER}]]);
  tex.runtoks('MyToks');
}
\directlua{
  luatexbase.remove_from_callback(
    'open_read_file',
    'CDR runtoks input'
  );
}

TEST without the callback
\textbf{BEFORE}
RAW LINE 1
RAW LINE 2
RAW LINE 3

\textbf{AFTER}
\directlua{
  texio.write_nl('From scantoks wihtout callbacks');
  tex.scantoks('MyToks',0,[[\string\textbf{BEFORE}\stringRAW LINE 1
RAW LINE 2
RAW LINE 3
\string\textbf{AFTER}]]);
  tex.runtoks('MyToks');
}
\section{with function}
\directlua{
  tex.forcehmode();
  luatexbase.add_to_callback(
    'open_read_file',
    function (file_name)
      texio.write_nl('');
      texio.write_nl('**** OPEN READ FILE '..file_name);
      local i = 0
      return {
        reader = function ()
          i = i+1;
          texio.write_nl('READER LINE '..i);
          if i < 5 then
            return 'READER LINE '..i
          end
        end
      }
    end,
    'CDR runtoks input'
  );
  tex.runtoks(function()
    tex.sprint([[\string\textbf{BEFORE}\stringRAW LINE 1
RAW LINE 2
RAW LINE 3
\string\textbf{AFTER}]])
  end);
  luatexbase.remove_from_callback(
    'open_read_file',
    'CDR runtoks input'
  );
}

\end{document}
