% !TeX program=lualatex
% !TEX encoding = UTF-8 Unicode

\documentclass{beamer}
\mode<presentation>
\usepackage{fontspec}
\usepackage{coder}
\usepackage{ifthen}
\usepackage{emoji}{}

\title{Presentation of package \textsf{coder}}

\subtitle
{Complement to the manual}

\author[Author, Another] % (optional, use only with lots of authors)
{Jérôme LAURENS}

\institute[Universities of Somewhere and Elsewhere] % (optional, but mostly needed)
{
  Institut de mathématiques de Bourgogne\\
  Université de Bourgogne, France
}

\date % (optional, should be abbreviation of conference name)
{\today}

% If you wish to uncover everything in a step-wise fashion, uncomment
% the following command: 

%\beamerdefaultoverlayspecification{<+->}


\begin{document}

\begin{frame}
  \titlepage
\end{frame}

\begin{frame}{Outline}
  \tableofcontents
  % You might wish to add the option [pausesections]
\end{frame}

\section{Motivation}

\subsection{The Basic Problem That We Studied}

\begin{CDRBlockSave}{Snippet1}
\textbf{Line 1}
\textbf{Line 2}
\textbf{Line 3}
\end{CDRBlockSave}
\makeatletter
\newcommand{\MyCDRNumberMainA}[1]{
  \setlength{\unitlength}{1cm}
  \begin{picture}(0,0)
  \put(-1,0){\llap{\ifthenelse{#1=1}{\emoji{t-rex}}
    {\ifthenelse{#1=2}{\emoji{penguin}}{\emoji{whale}}}%
  } }
  \end{picture}
  %
  \ifthenelse{#1=\beamer@slidenumber}{%
    \color{red}\bfseries}{}#1
}
\makeatother
\begin{CDRBlockSave}{SourceA}
\makeatletter
\newcommand{\MyCDRNumberMainA}[1]{
  \setlength{\unitlength}{1cm}
  \begin{picture}(0,0)
  \put(-1,0){\llap{\ifthenelse{#1=1}{\emoji{t-rex}}
    {\ifthenelse{#1=2}{\emoji{penguin}}{\emoji{whale}}}%
  } }
  \end{picture}
  %
  \ifthenelse{#1=\beamer@slidenumber}{%
    \color{red}\bfseries}{}#1
}
\makeatother
\end{CDRBlockSave}

\begin{frame}
{Playing with line numbers}
{CDRBlockSave on slide number \insertslidenumber}
\begin{block}{CDRBlockUse}
\renewcommand{\CDRNumberMain}{\MyCDRNumberMainA}
\CDRBlockUse[
  xleftmargin=2cm,
]{Snippet1}
\end{block}
\only<+(2)>{}
\end{frame}
\begin{frame}
{Playing with line numbers}
{SourceA}
\begin{block}{Source}
\CDRBlockUse{SourceA}
\end{block}
\end{frame}
\begin{frame}
{Playing with line numbers}
{SourceA}
\begin{block}{Source}
\CDRBlockUse{SourceA}
\end{block}
\end{frame}
\begin{frame}
{Playing with line numbers}
{SourceA}
\begin{block}{Source}
\CDRBlockExe{SourceA}
\end{block}
\end{frame}

\end{document}



\documentclass{beamer}

\usepackage{fontspec}
\usepackage{xcolor}
\usepackage{hyperref}
\hypersetup{
    colorlinks,
    linkcolor={red!50!black},
    citecolor={blue!50!black},
    urlcolor={blue!80!black}
}
\usepackage{float}
\usepackage{tcolorbox}
\usepackage{coder}
\usepackage{lipsum}
\usepackage{luacode} % JL
\usepackage{unicode-math}
\usepackage{multicol}
\usepackage{efbox}
\usepackage{mboxfill}
\usepackage{setspace}
\usepackage{relsize}


\ExplSyntaxOn

\cs_new_protected:Npn \cs #1 {
  \texttt{
    \textbackslash
    \tl_rescan:nn {
      \cctab_select:N \c_str_cctab
    } {
      #1
    }
  }
}
\cs_new:Npn \pkg { \textsf }

\cs_new:Npn \MyOption #1 {
  \texttt{#1}
}

\ExplSyntaxOff

\NewDocumentCommand \itemtt { o } {
  \IfNoValueTF { #1 } {
    \item
  } {
    \item[\ttfamily#1]
  }
}

\NewDocumentCommand \MyMeta {om} {%
  {\normalfont$⟨${\itshape#2}$\,⟩$%
  \IfValueT{#1}{\textsubscript{#1}}}%
}
\let\meta\MyMeta
\NewDocumentCommand \MyMetatt {om} {%
  {\normalfont$⟨${\ttfamily#2}$\,⟩$%
  \IfValueT{#1}{\textsubscript{\ttfamily#1}}}%
}
\let\metatt\MyMetatt

\def\CDRCheckRed {}
\def\CDRCheckGreen {}
\def\CDRCheckProhibited {}

\colorlet{CDROptions}{magenta!66!black}
\CDRSet{
  only top,
  tags/lua={
    lang=lua,
    style=autumn,
    numbers=right,
    show tags=mirror,
  },
  tags/python={
    lang=python,
    style=trac,
    numbers=left,
    show tags=mirror,
  },
  tags/latex|src|options={
    lang=tex,
    style=default,
    numbers=none,
    show tags=none,
    escapeinside=⟨⟩,
  },
  tags/options={
    format=\color{CDROptions}\bfseries,
  },
}

\newfontfamily\TGCFont{TeX Gyre Cursor}[NFSSFamily=TGC]

\begin{document}
\title{User manual for the \pkg{coder} package}
\author{Jérôme LAURENS\thanks{E-mail: jerome.laurens@u-bourgogne.fr}}
\maketitle
\tableofcontents

\end{document}


