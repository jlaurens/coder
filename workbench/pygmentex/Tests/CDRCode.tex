% !TeX program=lualatex
% !TeX root=../coder_test.tex


\subsection{Fancyvrb basics}

\ExplSyntaxOn
\group_begin:

\cs_new:Npn \CDRVerbA #1 {
  \DefineShortVerb{#1}
  \SaveVerb[aftersave={\UndefineShortVerb{#1}},]{Verb}#1
}


\CDRVerbA!ABCD!



ABCDE
\UseVerb{Verb}

\CDRVerbA|ABCD|

\UseVerb{Verb}
\makeatletter
\NewDocumentCommand\Profco{O{}m}{
  \DefineShortVerb{#2}
  \SaveVerb[#1,aftersave={
    \UndefineShortVerb{#2}
    \mbox{\FV@UseKeyValues\FV@FormattingPrep\FV@SV@Verb}
    }
  ]{Verb}#2
}
\makeatother
\Profco|BCDG|

\UseVerb{Verb}

\group_end:
\ExplSyntaxOff
\bgroup
\begin{Verbatim}[
%  fontshape=b,
  fontfamily=CDRVerbatimFont,%tt,%helvetica, courier
  fontseries=b,
  fontsize=\large,
  showspaces=true,
  showtabs=true,
  tabsize=4,
  reflabel=verb0
]
First verba	tim line.
Second verbatim line.
\end{Verbatim}
See the verbatim on page~\pageref{verb0}
\\
\DefineShortVerb{\?}%
YYY%
\SaveVerb{ABC}?A B C?%
\UndefineShortVerb{\?}
\fvset{
fontfamily=menlo,
fontshape=it,
fontseries=auto,
fontsize=\Large,
}%
\UseVerb[showspaces]{ABC}%
YYY
\CDRSet{pygments=false}
\DefineShortVerb{\|}%
\SaveVerb{ABCE}|YY YY|%
\UndefineShortVerb{\|}
\UseVerb{ABCE}%

\egroup

\subsection{Mimic}

\bgroup
\makeatletter
\ExplSyntaxOn
\def\CDRCode{\FV@Command{}{CDRCode}}
\begingroup
\catcode`\^^M=\active
\gdef\FVC@CDRCode#1{
  \begingroup
    \FV@UseKeyValues
    \FV@FormattingPrep
    \FV@CatCodes
    \outer\def^^M{}
    \catcode`#1=12
    \cs_set:Npn \CDR: {\def\FancyVerbGetVerb####1####2}
    \exp_last_unbraced:NV\CDR:\string#1{\typeout{DEBUG=====##2}\mbox{##2}\endgroup}
    \FancyVerbGetVerb\FV@EOL
}
\endgroup
\ExplSyntaxOff
\makeatother

\CDRCode*|A B C D|
\CDRCode|A B C D|

\egroup

\subsection{Coder}

\ExplSyntaxOn
\group_begin:
\ExplSyntaxOff

\CDRSet{cache=false,debug=true}
\CDRSet{pygments=false}
A\\
\CDRCode|<fontsize=\Large>|\\
B\\
\CDRCode[fontsize=\Large]|<fontsize=\Large>|\\
\CDRSet{pygments=true}
A\\
\CDRCode|<fontsize=\Large>|\\
B\\
\CDRCode[fontsize=\Large]|<fontsize=\Large>|\\

\CDRCode!<ABC DEF>!\\
1\\
\CDRCode[fontfamily=menlo]!<fontfamily=menlo>!\\
2\\
\CDRCode[showspaces=true]!<showspaces = true>!\\
\Verb[showspaces=true]!<showspaces = true>!\\
3\\
\CDRCode[fontsize=\Large]|<fontsize=\Large>|\\
4\\
\ExplSyntaxOn
\group_begin:
\cs_set:Npn \PackageError #1 #2 #3 {
\tl_to_str:n{\PackageError} #1\\
\exp_args:Ne \tl_to_str:n{#2}\\
\exp_args:Ne \tl_to_str:n{#3}\\
}
Expecting\space ERROR:\\
\CDRCode[engine=flower]|<engine=flower>|\\
\group_end:
5\\
\CDRCodeEngineNew{wave}{wavy(#2)}
\CDRCode[engine=wave]|<engine=wave>|\\
LAST\\
\CDRCode|<THIS IS THE LAST!!!>|\\
%
%\CDRCode|ABC DEF GHI|

\group_end:
\ExplSyntaxOff


\subsection{With pygments}

\begin{luacode}

tex.print('python=='..CDR.PYTHON_PATH)

local json = _ENV.utilities.json
local lfs   = _ENV.lfs

CDR.cache_clean_unused = function (self)
end

local load_exec = CDR.load_exec

CDR.Xload_exec = function (cmd)
print('CMD', cmd)
load_exec(cmd)
end
\end{luacode}


\CDRSet{python path=/usr/bin/python}
PATH==\directlua{tex.print(CDR.PYTHON_PATH)}\\
\CDRSet{python path=}
PATH==\directlua{tex.print(CDR.PYTHON_PATH)}\\

\ExplSyntaxOn

\CDR_has_pygments:TF {
  PYGMENTS~TEST\\
} {
  NO~PYGMENTS~TEST
  \ExplSyntaxOff
  \endinput
}

\group_begin:
\CDRSet{pygments=true}


\CDRSet{pygments=false}
AB CD=\CDRCode|AB CD|\\
\CDRSet{pygments=true}
AB CD=\CDRCode|AB CD\textbf{AB}|\\
AB CD=\CDRCode[fontfamily=menlo]|AB CD\textbf{AB}|\\
AB CD=\CDRCode[formatcom=\color{red}]|AB CD\textbf{AB}|\\
AB CD=\CDRCode[format=\color{blue}]|AB CD\textbf{AB}|\\
AB CD=\CDRCode[format=\color{blue},formatcom=\color{red}]|AB CD\textbf{AB}|\\
\group_end:
\ExplSyntaxOff
\CDRSet{pygments=true,cache=false}
AB CD=\CDRCode|AB CD\textbf{AB}| et la suite\\
AB CD=\CDRCode[style=default]|AB CD\textbf{AB}| et la suite\\
AB CD=\CDRCode[style=autumn]|AB CD\textbf{AB}| et la suite\\
\endinput

AB CD=\CDRCode[cache=true]|AB CD\textbf{AB}| et la suite\\
AB CD=\CDRCode[cache=true,debug=true]|AB CD\textbf{AB}| et la suite\\
\ExplSyntaxOn
\group_begin:
\ExplSyntaxOff
\CDRSet{tags/X/style=autumn}
style autumn  AB CD=\CDRCode[tag=X]|AB CD\textbf{AB}| et la suite\\
style default AB CD=\CDRCode[tag=Y]|AB CD\textbf{AB}| et la suite\\
style default AB CD=\CDRCode[engine=efbox, efbox engine options={hidealllines,backgroundcolor=red}, engine options={backgroundcolor=yellow}]|AB CD\textbf{AB}| et la suite\\
style default AB CD=\CDRCode[format=\color{blue}]|AB CD\textbf{AB}| et la suite\\
\ExplSyntaxOn
\group_end:
\ExplSyntaxOff
\CDRSet{pygments=false}

