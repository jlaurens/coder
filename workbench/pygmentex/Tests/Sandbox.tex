% !TeX program=lualatex
% !TeX root=../coder_test.tex

\def\baselinestretch{1}
\begin{CDRBlockSave}{BASE}
line 1
physical line 2
\physical line 3
line 4
line 5
\end{CDRBlockSave}
\CDRBlockUse[]{BASE}
\colorlet{blackenta}{blue}
\CDRBlockUse[
  firstnumber=10,
  line prefix={#1/},
  line postfix={/#2},
  line suffix={/#1/#2},
]{BASE}
\CDRBlockUse[
  firstnumber=10,
  line color=green!10!white,
  text color=magenta!50!black,
  pygments=true,
]{BASE}
\CDRBlockUse[
  firstnumber=10,
  line color=green!10!white,
  text color=blue,
  pygments=false,
  line depth=0.3
]{BASE}
\CDRBlockUse[
  firstnumber=10,
  line color=\ifnumodd{#1}{green!20!white}{green!10!white},
]{BASE}
\CDRBlockUse[
  firstnumber=10,
  line color=\ifnumodd{#2}{green!20!white}{green!10!white},
]{BASE}

\CDRDebugOn
\begin{CDRBlock}[
  line prefix=\ifnumequal{#1}{2}{%
    \tikzmark{2E}%
  }{},
  line postfix=\ifnumequal{#1}{3}{%
    \tikzmark{3E}%
  }{},
  line suffix=\ifnumequal{#1}{4}{%
    \tikzmark{4E}%
  }{},
]
line 1
line 2
line 3
line 4
line 5
\end{CDRBlock}
The prefix of line 2 goes here\tikzmark{2B},
the postfix of line 3 goes here\tikzmark{3B}
and the suffix of line 4 goes here\tikzmark{4B}.
\begin{tikzpicture}[remember picture]
\draw[overlay] (pic cs:2B) -- (pic cs:2E);
\draw[overlay] (pic cs:3B) -- (pic cs:3E);
\draw[overlay] (pic cs:4B) -- (pic cs:4E);
\end{tikzpicture}
\endinput

\CDRCodeSave{ABCD}|\textbf{XYZ}|
ABCDE
\CDRCodeUse{ABCD}
GHI
\CDRCodeUse[pygments=false]{ABCD}
\endinput

\noindent

\bgroup

\makeatletter
\ExplSyntaxOn

\prg_set_conditional:Nnn \CDR_if_TEST: { p, T, F, TF } { \prg_return_true: }

T==\CDR_if_TEST:TF TF\\


\tl_set:Nn \l_CDR_tl { \prg_return_true: }
\exp_args:NNnV
\prg_set_conditional:Nnn \CDR_if_TEST: { p, T, F, TF } \l_CDR_tl

T==\CDR_if_TEST:TF TF\\


\tl_set:Nn \l_CDR_tl { \prg_return_true: }
\tl_put_left:Nn \l_CDR_tl {
\prg_set_conditional:Nnn \CDR_if_TEST: { p, T, F, TF }
}
\l_CDR_tl
T==\CDR_if_TEST:TF TF\\


\def\FVB@TEST {
  \@bsphack
  \FV@VerbatimBegin
  \FV@Scan
}
\def\FVE@TEST {
  \FV@VerbatimEnd
  \CDR_if_TEST:T{\typeout{T}}
  \@esphack
}

  \@namedef{TEST}{\FV@Environment{}{TEST}}
  \@namedef{endTEST}{\FVE@TEST}



\ExplSyntaxOff
\makeatother


\begin{TEST}
A
\end{TEST}

\egroup

\end{document}