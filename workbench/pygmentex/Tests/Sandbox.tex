% !TeX program=lualatex
% !TeX root=../coder_test.tex

\def\baselinestretch{1}
\begin{CDRBlockSave}{BASE}
line 1
physical line 2
line 3
\end{CDRBlockSave}
\CDRBlockUse[]{BASE}
\RenewDocumentCommand \CDRLineContent{ mmm } {%$
#3\textcolor{red}{THIS IS LINE/#1/#2}}
\colorlet{blackenta}{blue}
\CDRBlockUse[
  firstnumber=10,
  line color=green!10!white,
  text color=magenta!50!black,
  pygments=true,
]{BASE}
\CDRBlockUse[
  firstnumber=10,
  line color=green!10!white,
  text color=blue,
  pygments=false,
]{BASE}
\endinput

\CDRCodeSave{ABCD}|\textbf{XYZ}|
ABCDE
\CDRCodeUse{ABCD}
GHI
\CDRCodeUse[pygments=false]{ABCD}
\endinput

\noindent

\bgroup

\makeatletter
\ExplSyntaxOn

\prg_set_conditional:Nnn \CDR_if_TEST: { p, T, F, TF } { \prg_return_true: }

T==\CDR_if_TEST:TF TF\\


\tl_set:Nn \l_CDR_tl { \prg_return_true: }
\exp_args:NNnV
\prg_set_conditional:Nnn \CDR_if_TEST: { p, T, F, TF } \l_CDR_tl

T==\CDR_if_TEST:TF TF\\


\tl_set:Nn \l_CDR_tl { \prg_return_true: }
\tl_put_left:Nn \l_CDR_tl {
\prg_set_conditional:Nnn \CDR_if_TEST: { p, T, F, TF }
}
\l_CDR_tl
T==\CDR_if_TEST:TF TF\\


\def\FVB@TEST {
  \@bsphack
  \FV@VerbatimBegin
  \FV@Scan
}
\def\FVE@TEST {
  \FV@VerbatimEnd
  \CDR_if_TEST:T{\typeout{T}}
  \@esphack
}

  \@namedef{TEST}{\FV@Environment{}{TEST}}
  \@namedef{endTEST}{\FVE@TEST}



\ExplSyntaxOff
\makeatother


\begin{TEST}
A
\end{TEST}

\egroup

\end{document}