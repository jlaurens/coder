% !TeX program=lualatex
% !TeX root=../coder_test.tex
\noindent

\ExplSyntaxOn
\CDR_int_new:cn { __last } { 0 }
\keys_define:nn { CDR@Line } {
  last .code:n = \CDR_int_set:cn { __last } { #1 },
}
\ExplSyntaxOff

%\bgroup
%
%%
%ESSAI
%\begin{Verbatim}
%[]
%ESSAI
%\end{Verbatim}
%
%\makeatletter
%\def\FVB@MyVerbatim{
%\FV@VerbatimBegin\FV@Scan
%}
%\def\FVE@MyVerbatim{
%\FV@VerbatimEnd
%}
%\makeatother
%\DefineVerbatimEnvironment{MyVerbatim}{MyVerbatim}{}
%
%
%\begin{MyVerbatim}[fontsize=\Large]
%ESSAI
%\end{MyVerbatim}
%
%
%\ExplSyntaxOn
%\makeatletter
%
%\cs_new:Npn \CDR_hilight_record:n #1 {
%}
%
%\def\CDR@ListProcessLine#1{%
%  \hbox to \hsize{#1
%}%
%}
%
%
%
%% \begin{variable}{\l_CDR_pyg_bool}
%%    \begin{MacroCode}[OK]
%\bool_new:N \l_CDR_pyg_bool
%\bool_set_false:N \l_CDR_pyg_bool
%%    \end{MacroCode}
%% \end{variable}
%%
%
%\prg_set_conditional:Nnn \CDR_tag_if_truthy:c { p, T,  F, TF } {
%  \exp_args:Ne
%  \str_compare:nNnTF {
%    \exp_args:Ne \str_lowercase:n { \CDR_tag_get:c { #1 } }
%  } = { false } {
%    \prg_return_false:
%  } {
%    \prg_return_true:
%  }
%}
%
%\cs_new:Npn \CDR@Line {
%  \peek_meaning_ignore_spaces:NTF [ { \CDR_line:nnn } { \CDR_line:nn }
%}
%\cs_new:Npn \CDR_line:nnn [ #1 ] {
%  \keys_set:nn { CDR@Line } { #1 }
%  \CDR_line:nn
%}
%\cs_new:Npn \CDR_line:nn #1 #2 {
%  \typeout{CDR@Line:\the\leftmargin/#1/}
%  \hbox to \hsize {
%    \kern \leftmargin
%    \CDR_info_left:n { #1 }
%    \hbox to \linewidth {
%      \FV@LeftListFrame
%      #2
%      \hss
%      \FV@RightListFrame
%    }
%  }
%}
%
%\cs_new:Npn \CDR@NumberFormat {
%  \CDR_tag_get:c { numbers~format }
%}
%\cs_new:Npn \CDR@TagsFormat {
%  \CDR_tag_get:c { tags~format }
%}
%\cs_new:Npn \CDR@NumberSep {
%  \hspace{ \CDR_tag_get:c { numbersep } }
%}
%
%%    \end{MacroCode}
%% \begin{function}{\FVB@CDRBlock}
%% \pkg{fancyvrb} helper to begin the environment.
%% \end{function}
%%    \begin{MacroCode}
%
%\ExplSyntaxOff
%
%\makeatother
%\begin{luacode}
%
%function CDR:cache_clean_unused()
%end
%
%\end{luacode}
%%
%\begin{CDRBlock}[
%  tags=,
%  fontfamily=menlo,
%  fontsize=\Large,
%  pygments=false,
%  lang=lua,
%  numbers=left,
%  frame=lines,
%  format=\color{green},
%  debug=true,
%  showspaces
%]
%local f = function(arg)
%  return arg ** arg
%end
%\end{CDRBlock}
%\typeout{IN PROGRESSSSSSSSSSSSSSSSSSSSSSSSSSSSSSSSSSSSSSSSSSSSSS}
%Next should not be void
%\CDRSet{cache=false}
%\begin{CDRBlock}[
%  tags=,
%  fontfamily=menlo,
%  fontsize=\large,
%  pygments=true,
%  lang=python,
%  numbers=left,
%  frame=single,
%  debug=true,
%]
%def foo(arg):
%  return arg ** arg
%\end{CDRBlock}
%
%Next should not be void
%\CDRSet{cache=false}
%\begin{CDRBlock}[
%  tags=,
%  fontfamily=menlo,
%  fontsize=\large,
%  pygments=true,
%  lang=lua,
%  numbers=left,
%  frame=lines,
%  debug=true,
%  frame=single,
%  framerule=1mm,
%  framesep=3mm,
%  rulecolor=\color{red},
%  fillcolor=\color{yellow},
%  showspaces,
%  baselinestretch=1.75
%]
%function foo(arg)
%  return arg ** arg
%end
%\end{CDRBlock}
%
%\egroup

\subsection{Line numbering}

\bgroup

\makeatletter
\ExplSyntaxOn


\cs_set:Npn \CDR@NumberFormat {
  \CDR_tag_get:c { numbers~format }
}
\cs_set:Npn \CDR@NumberSep {
  \hspace{ \CDR_tag_get:c { numbersep } }
}


%    \end{MacroCode}
%
% \begin{function}{\CDR_block_setup_tags:}
% Utility to setup the tags and the tag inheritance tree.
% \end{function}
%    \begin{MacroCode}
\cs_new:Npn \CDR_block_setup_tags: {
  \CDR_tag_if_exist_here:ccT { __local } { tags } {
    \CDR_tag_get:cN { tags } \l_CDR_clist
    \clist_if_empty:NF \l_CDR_clist {
      \clist_gset_eq:NN \g_CDR_tags_clist \l_CDR_clist
    }
  }
  \clist_if_empty:NT \g_CDR_tags_clist {
    \PackageWarning
      { coder }
      { No~(default)~tags~provided. }
  }
  \CDR_tag_inherit:cf { __local } {
    \g_CDR_tags_clist,
    __block, default.block, __pygments.block, __fancyvrb.block, __fancyvrb.number,
     __pygments, default, __fancyvrb,
  }
%    \end{MacroCode}
% Create a |\CDR_tags_if_visible:TF| raw conditional.
%    \begin{MacroCode}
  \exp_args:NNx
  \cs_set:Npn \CDR_tags_if_visible:TF {
    \bool_if:nTF { \CDR_tag_if_truthy_p:c { show~tags }
        && ( ! \CDR_tag_if_truthy_p:c { only~top }
          || ! \CDR_clist_if_eq_p:NN \g_CDR_tags_clist \g_CDR_last_tags_clist
        )
    } {
      \exp_not:n { \use_i:nn }
    } {
      \exp_not:n { \use_ii:nn }
    }
  }
%    \end{MacroCode}
% For each \metatt{tag name}, create an \pkg{l3int} variable and initialize it to 1.
%    \begin{MacroCode}
  \clist_map_inline:Nn \g_CDR_tags_clist {
    \CDR_int_if_exist:cF { ##1 } {
      \CDR_int_new:cn { ##1 } { 1 }
    }
  }
}
%    \end{MacroCode}
%
% \begin{function}{\FVB@CDRBlock}
% \pkg{fancyvrb} helper to begin the environment.
% \end{function}
%    \begin{MacroCode}
\def\FVB@CDRBlock {
  \@bsphack
  \group_begin:
  \prg_set_conditional:Nnn \CDR_if_block: { p, T, F, TF } {
    \prg_return_true:
  }  
  \CDR_tag_keys_set:nn { __block } { __initialize }
%    \end{MacroCode}
% Reading the options: we absorb the options available in |\FV@KeyValues|,
% first for % \pkg{l3keys} modules, then for |\fvset|.
%    \begin{MacroCode}
  \CDR_keys_inherit:Vnn \c_CDR_tag { __local } {
    __block, __pygments.block, default.block,
    __pygments, default,
  }
  \CDR_tag_keys_set_known:nVN { __local } \FV@KeyValues \l_CDR_kv_clist
  \CDR_tag_provide_from_kv:V \l_CDR_kv_clist
  \CDR_tag_keys_set_known:nVN { __local } \l_CDR_kv_clist \l_CDR_kv_clist
%    \end{MacroCode}
% By default, this code chunk will have the same list of tags
% as the last code block or last |\CDRExport| stored in |\g_CDR_tags_clist|.
% This can be overwritten with the |tags=...| user interface.
% At least one tag must be provided.
%    \begin{MacroCode}
  \CDR_block_setup_tags:
  \lua_now:n {
    CDR:hilight_block_setup('g_CDR_tags_clist')
  }
%    \end{MacroCode}
% |\l_CDR_pyg_bool| is |true| iff one of the tags needs \pkg{pygments} or there is no tag and |pygments=true| was given.
%    \begin{MacroCode}
  \bool_set_false:N \l_CDR_pyg_bool
  \bool_set:Nn \l_CDR_pyg_bool {
    \CDR_tag_if_truthy_p:c { pygments }
  }
%    \end{MacroCode}
% And we can setup the engine.
%    \begin{MacroCode}
  \CDR_tag_get:cN { engine } \l_CDR_engine_tl
  \CDR_if_code_engine:VF \l_CDR_engine_tl {
    \PackageError
      { coder }
      { \l_CDR_engine_tl\space block~engine~unknown,~replaced~by~'default' }
      {See~\CDRBlockEngineNew~in~the~coder~manual}
    \tl_set:Nn \l_CDR_engine_tl { default }
  }
  \CDR_tag_get:cN { \l_CDR_engine_tl~engine~options } \l_CDR_opts_tl
  \CDR_tag_get:cN { engine~options } \l_CDR_tl
  \tl_if_empty:NF \l_CDR_tl {
    \tl_put_right:Nn \l_CDR_opts_tl {,}
    \tl_put_right:NV \l_CDR_opts_tl \l_CDR_tl
  }
  \exp_args:NnV
  \use:c { \CDR_block_engine:V \l_CDR_engine_tl } \l_CDR_opts_tl
%    \end{MacroCode}
% Now we branch according whether \pkg{pygments} is used or not.
%    \begin{MacroCode}
  \bool_if:NTF \l_CDR_pyg_bool {
    \CDRBlock@Pyg
  } {
    \CDRBlock@FV
  }
  \FV@Scan
}
\def\FVE@CDRBlock {
  \bool_if:NTF \l_CDR_pyg_bool {
    \endCDRBlock@Pyg
  } {
    \endCDRBlock@FV
  }
  \use:c { end \CDR_block_engine:V \l_CDR_engine_tl }
  \group_end:
  \@esphack
}
\DefineVerbatimEnvironment{CDRBlock}{CDRBlock}{}


\newenvironment { CDRBlock@Pyg } {
%    \end{MacroCode}
% Catch the keys related to numbering before they are forwarded to \pkg{fancyvrb}.
%    \begin{MacroCode}
  \CDR_keys_inherit:Vnn \c_CDR_tag { __local } {
    __fancyvrb.number
  }
  \CDR_tag_keys_set_known:nVN { __local } \l_CDR_kv_clist \l_CDR_kv_clist
  \CDR_keys_inherit:Vnn \c_CDR_tag { __local } {
    __fancyvrb, __fancyvrb.block
  }
  \exp_args:NnV
  \CDR_tag_keys_set:nn { __local } \l_CDR_kv_clist
  \exp_args:NNV
  \def \FV@KeyValues \l_CDR_kv_clist
  \exp_args:NNnx
  \prg_set_conditional:Nnn \CDR_if_numbering: { p, T, F, TF } {
    \CDR_tag_if_eq:cnTF { numbers } { none } {
      \exp_not:n { \prg_return_false: }
    } {
      \exp_not:n { \prg_return_true: }
    }
  }
%    \end{MacroCode}
% Get the list of tags and setup \CDRLua{} for recording or hilighting.
%    \begin{MacroCode}
  \CDR_tag_get:cN {lang} \l_CDR_tl
  \lua_now:n { CDR:hilight_set_var('lang') }
  \CDR_tag_get:cN {cache} \l_CDR_tl
  \lua_now:n { CDR:hilight_set_var('cache') }
  \CDR_tag_get:cN {debug} \l_CDR_tl
  \lua_now:n { CDR:hilight_set_var('debug') }
  \CDR_tag_get:cN {style} \l_CDR_tl
  \lua_now:n { CDR:hilight_set_var('style') }
  \CDR@StyleIfExist { \l_CDR_tl } { } {
    \lua_now:n { CDR:hilight_source(true, false) }
    \input { \l_CDR_pyg_sty_tl }
  }
  \CDR@StyleUseTag
  \clist_map_inline:nn { firstnumber, stepnumber } {
    \CDR_tag_get:cN {##1} \l_CDR_tl
    \lua_now:n { CDR:hilight_set_var('##1') }
  }
  \CDR_tag_get:cN {stepnumber} \l_CDR_tl
  \lua_now:n { CDR:hilight_set_var('stepnumber') }
  \CDR_tag_if_truthy:cTF {no~export} {
    \clist_map_inline:nn { i, ii, iii, iv } {
      \cs_set:cpn { FV@ListProcessLine@ ##1 } ####1 {
        \tl_set:Nn \l_CDR_tl { ####1 }
        \lua_now:n { CDR:record_line('l_CDR_tl') }
      }
    }
  } {
    \clist_map_inline:nn { i, ii, iii, iv } {
      \cs_set:cpn { FV@ListProcessLine@ ##1 } ####1 {
        \tl_set:Nn \l_CDR_tl { ####1 }
        \lua_now:n { CDR:record_line('l_CDR_tl') }
      }
    }
  }
  \def\FV@ProcessLine ##1 {
    \tl_set:Nn \l_CDR_tl { ##1 }
    \lua_now:n { CDR:record_line('l_CDR_tl') }
  }
} {
  \CDR_if_numbering:T {
    \clist_map_inline:Nn \g_CDR_tags_clist {
      \CDR_int_if_exist:cF { ##1 } {
        \CDR_int_new:cn { ##1 } { 1 }
      }
    }
  }
  \CDR_tag_get:c { format }
  \fvset{ commandchars=\\\{\} }
  \CDR@DefineSp
  \FV@VerbatimBegin
  \lua_now:n { CDR:hilight_source(false, true) }
  \makeatletter
  \input{ \l_CDR_pyg_tex_tl }
  \makeatother
  \CDR_if_numbering:T {
    \exp_args:Nf
    \str_case:nnF { \CDR_tag_get:c { firstnumber } } {
      { last } {
        \clist_map_inline:Nn \g_CDR_tags_clist {
          \CDR_int_gadd:cc { ##1 } { __last }
        }
      }
      { auto } {
        \CDR_int_add:cn { __last } { 1 }
        \clist_map_inline:Nn \g_CDR_tags_clist {
          \CDR_int_gset:cc { ##1 } { __last }
        }
      }
    } {
      \CDR_int_add:cn { __last } { \CDR_tag_get:c { firstnumber } }
      \clist_map_inline:Nn \g_CDR_tags_clist {
        \CDR_int_gset:cc { ##1 } { __last }
      }
    }
  }
  \FV@VerbatimEnd
}

\newenvironment { CDRBlock@FV } {
%    \end{MacroCode}
% \pkg{pygments} is not used, \pkg{fancyvrb} features.
% We record the key value options about numbering.
%    \begin{MacroCode}
  \exp_args:NnV
  \use:c { \CDR_block_engine:V \l_CDR_engine_tl } \l_CDR_opts_tl
  \CDR_keys_inherit:Vnn \c_CDR_tag { __local } {
    __fancyvrb.number,
  }
  \CDR_tag_keys_set_known:nVN { __local } \l_CDR_kv_clist \l_CDR_clist
  \CDR_tag_inherit:cn { __local } {
    __fancyvrb.number,
    __block,
  }
  \CDR_tag_if_eq:cnTF { numbers } { none } {
    \tl_set:Nn \l_CDR_tl { \prg_return_false: }
  } {
    \tl_set:Nn \l_CDR_tl { \prg_return_true: }
  }
  \exp_args:NNnV
  \prg_set_conditional:Nnn \CDR_if_numbering: { p, T, F, TF } \l_CDR_tl  
  \CDR_tag_if_truthy:cF {no~export} {
    \clist_map_inline:nn { i, ii, iii, iv } {
      \cs_set:cpn { FV@ListProcessLine@ ##1 } ####1 {
        \tl_set:Nn \l_CDR_tl { ####1 }
        \lua_now:n { CDR:record_line('l_CDR_tl') }
        \use:c { CDR@ListProcessLine@ ##1 } { ####1 }
      }
    }
  }
%    \end{MacroCode}
% Prepare the counters. The |__| int starts with 0, which means that it is unused.
% If the |firstnumber| value is ``last'' then it is used to store the first number.
%    \begin{MacroCode}
  \CDR_int_set:cn { __ } { 0 }
  \CDR_if_numbering:T {
    \CDR_tag_if_eq:cnT { firstnumber } { last } {
      \clist_map_inline:Nn \g_CDR_tags_clist {
        \clist_map_break:n {
          \CDR_int_set:cc { __ } { ##1 }
          \clist_put_right:Nx \l_CDR_kv_clist {
            firstnumber = \CDR_int_use:c { ##1 }
          }
        }
      }
    }
  }
  \exp_args:NNV
  \def \FV@KeyValues \l_CDR_kv_clist
  \FV@VerbatimBegin
} {
  \FV@VerbatimEnd
  \CDR_if_numbering:T {
    \CDR_int_compare:cNnTF { __ } > 0 {
      \CDR_int_set:cn { __ } {
        \value{FancyVerbLine} - \CDR_int_use:c { __ } + 1
      }
      \clist_map_inline:Nn \g_CDR_tags_clist {
        \CDR_int_gadd:cc { ##1 } { __ }
      }
    } {
      \CDR_int_set:cn { __ } { \value{FancyVerbLine} + 1 }
      \clist_map_inline:Nn \g_CDR_tags_clist {
        \CDR_int_gset:cc { ##1 } { __ }
      }
    }
  }
}

\prg_new_conditional:Nnn \CDR_clist_if_eq:NN { p, T, F, TF } {
  \tl_if_eq:NNTF #1 #2 {
    \prg_return_true:
  } {
    \prg_return_false:
  }
}

\ExplSyntaxOff
\makeatother

\subsubsection{\textsf{fancyvrb} linear}

\CDRSet{pygments=false}

\begin{Verbatim} [
  numbers=none,
]
A
\end{Verbatim}

\begin{CDRBlock} [
  tags=none,
  numbers=none,
]
A
\end{CDRBlock}
\ExplSyntaxOn
\typeout{none = \CDR_int_use:c { none }}
\ExplSyntaxOff

\begin{CDRBlock} [
  tags=none,
  numbers=left,
  firstnumber=last,
]
A
B
C
\end{CDRBlock}
\ExplSyntaxOn
\typeout{none = \CDR_int_use:c { none }}
\ExplSyntaxOff

\begin{CDRBlock} [
  tags=none,
  numbers=left,
  firstnumber=last,
]
A
B
C
\end{CDRBlock}
\ExplSyntaxOn
\typeout{none = \CDR_int_use:c { none }}
\ExplSyntaxOff

\begin{CDRBlock} [
  tags=none,
  numbers=left,
  firstnumber=last,
]
A
B
C
\end{CDRBlock}
\ExplSyntaxOn
\typeout{none = \CDR_int_use:c { none }}
\ExplSyntaxOff

\subsubsection{\textsf{fancyvrb} multi tags}

\CDRSet{pygments=false}

\begin{CDRBlock} [
  tags={A,B,C},
  numbers=left,
  firstnumber=last,
]
A=*1,B=1,C=1
\end{CDRBlock}

\begin{CDRBlock} [
  tags={B,C},
  numbers=left,
  firstnumber=last,
]
A=2,B=*2,C=2
A=2,B=*3,C=3
\end{CDRBlock}

\begin{CDRBlock} [
  tags=C,
  numbers=left,
  firstnumber=last,
]
A=2,B=4,C=*4
A=2,B=4,C=*5
\end{CDRBlock}

\begin{CDRBlock} [
  tags={C, B},
  numbers=left,
  firstnumber=last,
]
A=2,B=4,C=*6
A=2,B=5,C=*7
A=2,B=6,C=*8
\end{CDRBlock}

\begin{CDRBlock} [
  tags={B, A},
  numbers=left,
  firstnumber=last,
]
A=2,B=*7,C=9
A=3,B=*8,C=9
A=4,B=*9,C=9
\end{CDRBlock}
\begin{CDRBlock} [
  tags={A,C},
  numbers=left,
  firstnumber=last,
]
A=*5,B=10,C=9
A=*6,B=10,C=10
A=*7,B=10,C=11
\end{CDRBlock}

\ExplSyntaxOn
\CDR_int_compare:cNnF { A } = 8 { FAILED \\ }
\CDR_int_compare:cNnF { B } = {10} { FAILED \\ }
\CDR_int_compare:cNnF { C } = {12} { FAILED \\ }
\ExplSyntaxOff

\subsection{\textsf{fancyvrb} properties}

\begin{CDRBlock} [
  tags=none,
  numbers=left,
  firstnumber=last,
]
A
B
C
\end{CDRBlock}

\newpage

\subsection{Display tags}

\makeatletter
\ExplSyntaxOn

\cs_undefine:N \CDR@Line
\cs_undefine:N \CDR_line:nnn
\cs_undefine:N \CDR_line:nn
\cs_undefine:N \CDR@NumberFormat
\cs_undefine:N \CDR@TagsFormat
\cs_undefine:N \CDR@NumberSep
\cs_undefine:N \CDR_info:n

\cs_new:Npn \CDR@Line {
  \peek_meaning_ignore_spaces:NTF [ { \CDR_line:nnn } { \CDR_line:nn }
}
%    \end{MacroCode}
% \begin{function}{\CDR_line:nnn}
% \begin{syntax}
% \cs{CDR_line:nnn} \Arg{CDR@Line kv list} \Arg{line number} \Arg{line content}
% \end{syntax}
% This is the very first command called when typesetting.
% Some setup are made for line numbering, in particular the |\CDR_int_if_visible:n...|
% family is set here.
% The first line of the |...pyg.tex| files must read |\CDR@Line[last=...]{1}{...}|.
% \end{function}
%    \begin{MacroCode}
\CDR_int_new:cn { __start } { 0 }
\CDR_int_new:cn { __step }  { 0 }

\cs_new:Npn \CDR_line:nnn [ #1 ] {
  \keys_set:nn { CDR@Line } { #1 }
  \typeout{CDR_line:nnn:\CDR_int_use:c { __last }, #1 }
  \CDR_int_set:cn { __ } { 0 }
  \CDR_if_numbering:T {
    \CDR_tag_if_eq:cnTF { firstnumber } { last } {
      \clist_map_inline:Nn \g_CDR_tags_clist {
        \clist_map_break:n {
          \CDR_int_set:cc { __start } { ##1 }
        }
      }
    } {
      \CDR_tag_if_eq:cnTF { firstnumber } { auto } {
        \CDR_int_set:cn { __start } { 1 }
      } {
        \CDR_int_set:cn { __start } { \CDR_tag_get:c { firstnumber } }
      }
    }
    \CDR_int_set:cn { __step } { \CDR_tag_get:c { stepnumber } }
%    \end{MacroCode}
% \begin{function}[EXP,pTF]{\CDR_int_if_visible:n}
% \begin{syntax}
% \cs{CDR_int_if_visible:nTF} \Arg{line number} \Arg{true code} \Arg{false code}
% \end{syntax}
% Execute \metatt{true code} if the \metatt{linenumber} is visible, \metatt{false code} otherwise.
% The \metatt{linenumber} visibility depends on the value relative to first number and the step.
% This is relavant only when line numbering is enabled.
% Some setup are made for line numbering, in particular the |\CDR_int_if_visible:n...|.
% family is set here.
% \end{function}
%    \begin{MacroCode}
    \int_compare:nNnTF { \CDR_int:c { __step } } < 2 {
      \tl_set:Nn \l_CDR_tl { \prg_return_true: }
      \CDR_int_set:cn { __step } { 1 }
    } {
      \tl_set:Nn \l_CDR_tl { \int_compare:nNnTF {
          ( ##1 + \CDR_int:c { __start } - 1 )
          / \CDR_int:c { __step }  * \CDR_int:c { __step }
          - \CDR_int:c { __start } + 1
        } = { ##1 } {
          \prg_return_true:
        } {
          \prg_return_false:
        }
      }
    }
    \exp_args:NNnV
    \prg_set_conditional:Nnn \CDR_int_if_visible:n { p, T, F, TF } \l_CDR_tl
%    \end{MacroCode}
%    \begin{MacroCode}
  }
%    \end{MacroCode}
% \begin{function}{\CDR_if_middle_column:, \CDR_if_right_column:}
% \begin{syntax}
% \cs{CDR_int_if_middle_column:TF} \Arg{true code} \Arg{false code}
% \cs{CDR_int_if_right_column:TF} \Arg{true code} \Arg{false code}
% \end{syntax}
% Execute \metatt{true code} when in the middle or right column,
% \metatt{false code} otherwise.
% \end{function}
%    \begin{MacroCode}
  \prg_set_conditional:Nnn \CDR_if_middle_column: { p, T, F, TF } { \prg_return_false: }
  \prg_set_conditional:Nnn \CDR_if_right_column:  { p, T, F, TF } { \prg_return_false: }
  \exp_args:NNnx
  \prg_set_conditional:Nnn \CDR_if_single:  { p, T, F, TF } {
    \CDR_int_compare:cNnTF { __last } = 1 {
      \exp_not:n { \prg_return_true: }
    } {
      \exp_not:n { \prg_return_false: }
    }
  }
  \CDR_line:nn
}
%    \end{MacroCode}
% \begin{function}{\CDR_line_box_LR:nnn, \CDR_line_box_L:nn, \CDR_line_box_R:nn, \CDR_line_box:nn}
% \begin{syntax}
% \cs{CDR_line_box_LR:nnn} \Arg{left info} \Arg{line content} \Arg{right info}
% \cs{CDR_line_box_L:nn} \Arg{left info} \Arg{line content}
% \cs{CDR_line_box_R:nn} \Arg{right info} \Arg{line content}
% \end{syntax}
% Returns an hbox with the given material.
% The first |LR| command is the reference, from which are derived the |L|, |R| and |N| commands.
% At run time the |\CDR_line_box:nn| is defined to call one of the above commands
% (with the same signarture).
% \end{function}
%    \begin{MacroCode}
\cs_new:Npn \CDR_line_box_LR:nnn #1 #2 #3 {
  \hbox to \hsize {
    \kern \leftmargin
    #1
    \hbox to \linewidth {
      \FV@LeftListFrame
      #2
      \hss
      \FV@RightListFrame
    }
    #3
  }
}
\cs_new:Npn \CDR_line_box_L:nn #1 #2 {
  \CDR_line_box_LR:nnn { #1 } { #2 } {}
}
\cs_new:Npn \CDR_line_box_R:nn #1 #2 {
  \CDR_line_box_LR:nnn { } {#2} { #1 }
}
\cs_new:Npn \CDR_line_box_N:nn #1 #2 {
  \CDR_line_box_LR:nnn { } { #2 } {}
}
%    \end{MacroCode}
%
% \begin{function}{\CDR_line_box:NNn}
% \begin{syntax}
% \cs{CDR_line_box:NNnn} \meta{line boxer} \meta{info builder} \Arg{line number} \Arg{line content}
% \end{syntax}
% \end{function}
%    \begin{MacroCode}[OK]
\cs_new:Npn \CDR_line_build_N:nn {
   \CDR_line_box_N:nn
}
%    \end{MacroCode}
%    \begin{MacroCode}
\cs_new:Npn \CDR_line_build_L:nn #1 {
  \CDR_if_single:TF {
    \CDR_line_box_L:nn { \CDR_info_left_core:n { \CDR@NumberMain{ #1 } } }
  } {
    \CDR_int_compare:cNnTF { __start } > {
      \CDR_int:c { __last } / \CDR_int:c { __step } * \CDR_int:c { __step }
    } {
      \cs_set:Npn \CDR@Line ##1 {
        \CDR_line_box_L:nn { \CDR_info_left_core:n { \CDR@NumberOther{ ##1 } } }
      }
      \CDR_line_box_L:nn { \CDR_info_left_core:n { \CDR@NumberMain{ #1 } } }
    } {
      \cs_set:Npn \CDR@Line ##1 {
        \CDR_line_box_L:nn { \CDR_info_left_core:n { \CDR_number_alt:n { ##1 } } }
      }
      \CDR@Line { #1 }
    }
  }
}
%    \end{MacroCode}
%    \begin{MacroCode}
\cs_new:Npn \CDR_line_build_R:nn #1 {
  \CDR_if_single:TF {
    \CDR_line_box_R:nn { \CDR_info_right_core:n { \CDR@NumberMain{ #1 } } }
  } {
    \CDR_int_compare:cNnTF { __start } > {
      \CDR_int:c { __last } / \CDR_int:c { __step } * \CDR_int:c { __step }
    } {
      \cs_set:Npn \CDR@Line ##1 {
        \CDR_line_box_R:nn { \CDR_info_right_core:n { \CDR@NumberOther{ ##1 } } }
      }
      \CDR_line_box_R:nn { \CDR_info_right_core:n { \CDR@NumberMain{ #1 } } }
    } {
      \cs_set:Npn \CDR@Line ##1 {
        \CDR_line_box_R:nn { \CDR_info_right_core:n { \CDR_number_alt:n { ##1 } } }
      }
      \CDR@Line { #1 }
    }
  }
}
%    \end{MacroCode}
%
% \begin{function}{\CDR_line:nn}
% \begin{syntax}
% \cs{CDR_line:nn} \Arg{line number} \Arg{line content}
% \end{syntax}
% This is a one shot function called at the top of each code block.
% \end{function}
%    \begin{MacroCode}

\cs_new:Npn \CDR_line:nn {
  \exp_args:Nf
  \str_case:nnF { \CDR_tag_get:c { numbers } } {
    { left } {
      \CDR_tags_if_visible:TF {
        \CDR_line_build_L_T:nn
      } {
        \cs_set:Npn \CDR@Line { \CDR_line_build_L:nn }
      }
    }
    { right } {
      \CDR_tags_if_visible:TF {
        \CDR_line_box_R_T:nn
      } {
        \cs_set:Npn \CDR@Line { \CDR_line_build_R:nn }
      }
    }
    { none } {
       \cs_set:Npn \CDR@Line { \CDR_line_build_N:nn }
    }
  } { \PackageError { coder } { Unknown~numbers~options~:~ \CDR_tag_get:c { numbers }} }
  \CDR@Line
%  \hbox to \hsize {
%    \kern \leftmargin
%    \CDR_info_left:n { #1 }
%    \hbox to \linewidth {
%      \FV@LeftListFrame
%      #2
%      \hss
%      \FV@RightListFrame
%    }
%    \CDR_info_right:n { #1 }
%  }
}
%    \end{MacroCode}
% \begin{function}{\CDR_line:nn}
% \begin{syntax}
% \cs{CDR_line:nn} \Arg{line number} \Arg{line content}
% \end{syntax}
% Execute \metatt{true code} if the \metatt{linenumber} is visible, \metatt{false code} otherwise.
% The \metatt{linenumber} visibility depends on the value relative to first number and the step.
% This is relavant only when line numbering is enabled.
% Some setup are made for line numbering, in particular the |\CDR_int_if_visible:n...|.
% family is set here.
% \end{function}
%    \begin{MacroCode}
\cs_new:Npn \CDR_line_left:nn #1 #2 {
  \typeout{CDR@Line:\the\leftmargin/#1/}
  \hbox to \hsize {
    \kern \leftmargin
    \CDR_info_left:n { #1 }
    \hbox to \linewidth {
      \FV@LeftListFrame
      #2
      \hss
      \FV@RightListFrame
    }
    \CDR_info_right:n { #1 }
  }
}

\cs_new:Npn \CDR_info_left:n {
  \CDR_info_left_core:n
}


%    \end{MacroCode}
% \begin{function}{\CDR_info_left_core:n}
% \begin{syntax}
% \cs{CDR_info_left_core:n} \Arg{line number}
% \end{syntax}
% Core method to display the left info.
% \end{function}
%    \begin{MacroCode}
\cs_new:Npn \CDR_info_left_core:n #1 {
  \hbox_overlap_left:n {
    \cs_set:Npn \baselinestretch { 1 }
    { \CDR@NumberFormat
      #1
    }
    \CDR@NumberSep
  }
}
\cs_new:Npn \CDR_info_right_core:n #1 {
  \hbox_overlap_right:n {
    \cs_set:Npn \baselinestretch { 1 }
    \CDR@NumberSep
    { \CDR@NumberFormat
      #1
    }
  }
}

\cs_new:Npn \CDR_info_right:n {
  \use_none:n
}

\cs_new:Npn \CDR@NumberFormat {
  \CDR_tag_get:c { numbers~format }
}
\cs_new:Npn \CDR@TagsFormat {
  \CDR_tag_get:c { tags~format }
}
\cs_new:Npn \CDR@NumberSep {
  \hspace{ \CDR_tag_get:c { numbersep } }
}

%\cs_set:Npn \CDR_info_body:n { \use_none:n }

%    \end{MacroCode}
% \begin{function}{\CDR_info_body_core:n}
% First line.
% \end{function}
%    \begin{MacroCode}
%    \end{MacroCode}
% \begin{function}{\CDR_number_alt:n}
% First line.
% \end{function}
%    \begin{MacroCode}
\cs_set:Npn \CDR_number_alt:n #1 {
  \use:c { CDR@Number 
    \CDR_int_if_visible:nTF { #1 } { Main } { Other }
  } { #1 }
}

%    \end{MacroCode}
% \begin{function}{\CDRNumberMain, \CDR@NumberMain, \CDRNumberOther, \CDR@NumberOther}
% \begin{syntax}
% \cs{CDRNumberMain} \Arg{integer expression}
% \cs{CDRNumberOther} \Arg{integer expression}
% \end{syntax}
% The |@| commands are wrappers over the public versions.
% This is used when typesseting line numbers.
% The default |...Other...| function just gobble one argument.
% The line number is given relative to the |__start| integer value.
% \end{function}
%    \begin{MacroCode}
\cs_new:Npn \CDRNumberMain {
		\use:n
}
\cs_new:Npn \CDR@NumberMain #1 {
		\CDRNumberMain { \int_eval:n { #1 + \CDR_int:c { __start } - 1 } }
}
\cs_new:Npn \CDRNumberOther {
		\use_none:n
}
\cs_new:Npn \CDR@NumberOther #1 {
		\CDRNumberOther { \int_eval:n { #1 + \CDR_int:c { __start } - 1 } }
}
%    \end{MacroCode}
% \begin{function}{\CDR_info_body_tags:NN}
% \begin{syntax}
% \cs{CDR_info_body_tags:NN} \meta{info command:n} \meta{tags formatter:nn}
% \end{syntax}
% This is called exactly once for very first line of each pygmented source.
% \end{function}
%    \begin{MacroCode}
\cs_new:Npn \CDR_info_body_tags:NN #1 #2 {
  \CDR_if_numbering:TF {
    \CDR_int_compare:cNnTF { __last } = 1 {
      #2 {~} { \CDR@NumberMain { 1 } }
    } {
      \CDR_int_compare:cNnTF { __step } > 1 {
        \cs_set:Npn #1 ##1 {
          \CDR_number_alt:n
        }
        \CDR_int_if_visible:nTF { 2 } {
          #2 {} {}
        } {
          \CDR_int_compare:cNnTF { __last } > { 2 } {
            \CDR_int_if_visible:nTF { 3 } {
              #2 {} {}
            } {
              #2 {~} { \CDR@NumberMain { 1 } }
            }
          } {
            #2 {} {}
            \cs_set:Npn #1 ##1 {
              \CDR@NumberMain
            }
          }
        }
      } {
      \typeout{HEREEEEEEEE..................}
        \cs_set:Npn #1 {
          \CDR@NumberMain
        }
        #2 {} {}
      }
    }
  } {
    #2 {} {}
    \cs_set:Npn #1 ##1 { \use_none:n }
  }
}
%    \end{MacroCode}
% \begin{function}{\CDR_info_body_tags:N}
% \begin{syntax}
% \cs{CDR_info_body_tags:N} \meta{info command:n}
% \end{syntax}
% This is called exactly once for very first line of each pygmented source.
% \end{function}
%    \begin{MacroCode}
\cs_set:Npn \CDR_info_body_no_tags:N #1 {
    \CDR_if_numbering:TF {
      \CDR_int_compare:cNnTF { __last } = 1 {
        \CDR@NumberMain{ 1 }
      } {
        \cs_set:Npn #1 ##1 {
          \CDR_number_alt:n
        }
        \CDR_int_compare:cNnTF { __step } = 2 {
          \CDR_number_alt:n { 1 }
        } {
          \CDR@NumberMain{ 1 }
          \CDR_int_compare:cNnT { __step } > 2 {
            \CDR_int_compare:cNnT { __last } > { 2 } {
              \CDR_int_if_visible:nF { 2 } {
                \cs_set:Npn #1 ##1 {
                  \cs_set:Npn #1 {
                    \CDR_number_alt:n
                  }
                  \CDR@NumberOther
                }
              }
            }
          }
        }
      }
    } {
      \cs_set:Npn #1 ##1 { \use_none:n }
    }
}
%    \end{MacroCode}
% \begin{function}{\CDR_info_body_a:n}
% \begin{syntax}
% \cs{CDR_info_left_body:n} \Arg{line number}
% \end{syntax}
% First line.
% \end{function}
% This a one shot function.
%    \begin{MacroCode}
\cs_set:Npn \CDR_info_left_body:n #1 {
  \cs_set:Npn \CDR_info_left_body:n #1 {
    \typeout{THIS IS AN ERROR}
  }
  \CDR_tags_if_visible:TF {
    \clist_gset_eq:NN \g_CDR_last_tags_clist \g_CDR_tags_clist
    \cs_set:Npn \CDR_tags_formatter:nn ##1 ##2 {
      \smash{
        \parbox[b]{\marginparwidth}{
          \raggedleft
          { \CDR@TagsFormat \g_CDR_tags_clist :} ##1 ##2
        }
      }
    }
    \CDR_info_body_tags:NN \CDR_info_left_body:n \CDR_tags_formatter:nn
  } {
    \CDR_info_body_no_tags:N \CDR_info_left_body:n
  }
}
\cs_set:Npn \CDR_info_right_body:n #1 {
  \CDR_tags_if_visible:TF {
    \clist_gset_eq:NN \g_CDR_last_tags_clist \g_CDR_tags_clist
    \smash{
      \parbox[b]{\marginparwidth}{
        \raggedright
        \CDR_info_body_tags:n
        { \CDR@TagsFormat :{~}\g_CDR_tags_clist}
      }
    }
  } {
    \CDR_info_body_no_tags:
  }
}
%    \end{MacroCode}
%    \begin{MacroCode}

%    \end{MacroCode}
% Display the line number unless the step is >0 and the next line number is displayed.
%    \begin{MacroCode}

\ExplSyntaxOff
\makeatother

tests start here:

%\begin{CDRBlock}[
%  tags={we are the champions,  whatever it isp},
%  fontfamily=menlo,
%  fontsize=\large,
%  pygments=true,
%  lang=lua,
%  numbers=left,
%  debug=true,
%  show tags
%]
%function foo(arg) return arg ** arg end -- tags expected
%\end{CDRBlock}

\ExplSyntaxOn
PYGMENTS: \CDR_tag_get:cc { __pygments } { pygments } \\
\ExplSyntaxOff
\CDRSet{
  tags={NONE,ENON},
  pygments=true,
  lang=lua,
  numbers=right,
  debug=true,
  show tags = false,
  firstnumber = 3,
}
\ExplSyntaxOn
PYGMENTS: \CDR_tag_get:cc { __pygments } { pygments } \\
TAGS: \CDR_tag_get:cc { default.block } { tags } \\
TAGS: \g_CDR_tags_clist \\
\ExplSyntaxOff
\begin{CDRBlock}[
  stepnumber=1,
]
function foo(arg) return 0 end
\end{CDRBlock}
\ExplSyntaxOn
NONE: \CDR_int_use:c { NONE } \\
ENON: \CDR_int_use:c { ENON } \\
\ExplSyntaxOff

\begin{CDRBlock}[
  stepnumber=1,
  firstnumber = last,
]
function foo(arg) -- no tags expected
  2
  3
  4
  5
  6
end
\end{CDRBlock}
\ExplSyntaxOn
NONE: \CDR_int_use:c { NONE } \\
ENON: \CDR_int_use:c { ENON } \\
\ExplSyntaxOff

\begin{CDRBlock}[
  stepnumber=1,
  firstnumber = auto,
]
function foo(arg) -- no tags expected
  2
  3
  4
  5
  6
end
\end{CDRBlock}
\ExplSyntaxOn
NONE: \CDR_int_use:c { NONE } \\
ENON: \CDR_int_use:c { ENON } \\
\ExplSyntaxOff

\egroup
\end{document}
\begin{CDRBlock}[
%  tags={we are the champions,  whatever it isp},
  fontfamily=menlo,
  fontsize=\large,
  pygments=true,
  lang=lua,
  numbers=left,
  debug=true,
  show tags,
  only top=false
]
function foo(arg) return arg ** arg end -- tags expected
\end{CDRBlock}
\begin{CDRBlock}[
  tags={whatever it isp,we are the champions},
  fontfamily=menlo,
  fontsize=\large,
  pygments=true,
  lang=lua,
  numbers=left,
  debug=true,
  show tags,
  only top
]
function foo(arg) return arg ** arg end -- tags expected
\end{CDRBlock}
\begin{CDRBlock}[
%  tags={we are the champions,  whatever it isp},
  pygments,
  lang=lua,
  numbers=none,
  debug,
  showspaces,
  show tags=false,
  only top=false,
]
function foo(arg) -- no tags expected
  -- second line
  -- third line
  -- fourth line
  -- fifth line
  return arg ** arg
end
\end{CDRBlock}

\egroup

\CDRSet{pygments=false}
