% !TeX program=lualatex
% !TeX root=../coder_test.tex
\noindent

\ExplSyntaxOn
\CDR_int_new:cn { __last } { 0 }
\keys_define:nn { CDR@Line } {
  last .code:n = \CDR_int_set:cn { __last } { #1 },
}
\ExplSyntaxOff

\bgroup

%
ESSAI
\begin{Verbatim}
[]
ESSAI
\end{Verbatim}

\makeatletter
\def\FVB@MyVerbatim{
\FV@VerbatimBegin\FV@Scan
}
\def\FVE@MyVerbatim{
\FV@VerbatimEnd
}
\makeatother
\DefineVerbatimEnvironment{MyVerbatim}{MyVerbatim}{}


\begin{MyVerbatim}[fontsize=\Large]
ESSAI
\end{MyVerbatim}


\ExplSyntaxOn
\makeatletter

\cs_new:Npn \CDR_hilight_record:n #1 {
}

\def\CDR@ListProcessLine#1{%
  \hbox to \hsize{#1
}%
}



% \begin{variable}{\l_CDR_pyg_bool}
%    \begin{MacroCode}[OK]
\bool_new:N \l_CDR_pyg_bool
\bool_set_false:N \l_CDR_pyg_bool
%    \end{MacroCode}
% \end{variable}
%

\prg_set_conditional:Nnn \CDR_tag_if_truthy:c { p, T,  F, TF } {
  \exp_args:Ne
  \str_compare:nNnTF {
    \exp_args:Ne \str_lowercase:n { \CDR_tag_get:c { #1 } }
  } = { false } {
    \prg_return_false:
  } {
    \prg_return_true:
  }
}

\def\CDR@Total#1{\typeout{TOTAL:\space#1}}

\cs_new:Npn \CDR@Line {
  \peek_meaning_ignore_spaces:NTF [ { \CDR_line:nnn } { \CDR_line:nn }
}
\cs_new:Npn \CDR_line:nnn [ #1 ] {
  \keys_set:nn { CDR@Line } { #1 }
  \CDR_line:nn
}
\cs_new:Npn \CDR_line:nn #1 #2 {
  \typeout{CDR@Line:\the\leftmargin/#1/}
  \hbox to \hsize {
    \kern \leftmargin
    \CDR_info:n { #1 }
    \hbox to \linewidth {
      \FV@LeftListFrame
      #2
      \hss
      \FV@RightListFrame
    }
  }
}

\cs_new:Npn \CDR@NumberFormat {
  \CDR_tag_get:c { numbers~format }
}
\cs_new:Npn \CDR@TagsFormat {
  \CDR_tag_get:c { tags~format }
}
\cs_new:Npn \CDR@NumberSep {
  \hspace{ \CDR_tag_get:c { numbersep } }
}

\cs_new:Npn \CDR_info:n #1 {
  \hbox_overlap_left:n {
    { \CDR@NumberFormat #1 }
    \CDR@NumberSep
  }
}

%    \end{MacroCode}
% \begin{function}{\FVB@CDRBlock}
% \pkg{fancyvrb} helper to begin the environment.
% \end{function}
%    \begin{MacroCode}

\ExplSyntaxOff

\makeatother
\begin{luacode}

function CDR:cache_clean_unused()
end

\end{luacode}
%
\begin{CDRBlock}[
  tags=,
  fontfamily=menlo,
  fontsize=\Large,
  pygments=false,
  lang=lua,
  numbers=left,
  frame=lines,
  format=\color{green},
  debug=true,
  showspaces
]
local f = function(arg)
  return arg ** arg
end
\end{CDRBlock}
\typeout{IN PROGRESSSSSSSSSSSSSSSSSSSSSSSSSSSSSSSSSSSSSSSSSSSSSS}
Next should not be void
\CDRSet{cache=false}
\begin{CDRBlock}[
  tags=,
  fontfamily=menlo,
  fontsize=\large,
  pygments=true,
  lang=python,
  numbers=left,
  frame=single,
  debug=true,
]
def foo(arg):
  return arg ** arg
\end{CDRBlock}

Next should not be void
\CDRSet{cache=false}
\begin{CDRBlock}[
  tags=,
  fontfamily=menlo,
  fontsize=\large,
  pygments=true,
  lang=lua,
  numbers=left,
  frame=lines,
  debug=true,
  frame=single,
  framerule=1mm,
  framesep=3mm,
  rulecolor=\color{red},
  fillcolor=\color{yellow},
  showspaces,
  baselinestretch=1.75
]
function foo(arg)
  return arg ** arg
end
\end{CDRBlock}

\egroup

\subsection{Line numbering}

\bgroup

\makeatletter
\ExplSyntaxOn


\cs_set:Npn \CDR@NumberFormat {
  \CDR_tag_get:c { numbers~format }
}
\cs_set:Npn \CDR@NumberSep {
  \hspace{ \CDR_tag_get:c { numbersep } }
}

\cs_set:Npn \CDR_info:n #1 {
  \hbox_overlap_left:n {
    { \CDR@NumberFormat #1/\CDR_int_use:c { __last } }
    \CDR@NumberSep
  }
}

%    \end{MacroCode}
%
% \begin{function}{\CDR_block_setup_tags:}
% Utility to setup the tags and the tag inheritance tree.
% \end{function}
%    \begin{MacroCode}
\cs_new:Npn \CDR_block_setup_tags: {
  \CDR_tag_if_exist_here:ccT { __local } { tags } {
    \CDR_tag_get:cN { tags } \l_CDR_clist
    \clist_if_empty:NF \l_CDR_clist {
      \clist_gset_eq:NN \g_CDR_tags_clist \l_CDR_clist
    }
  }
  \clist_if_empty:NT \g_CDR_tags_clist {
    \PackageWarning
      { coder }
      { No~(default)~tags~provided. }
  }
  \CDR_tag_inherit:cf { __local } {
    \g_CDR_tags_clist,
    __block, default.block, __pygments.block, __fancyvrb.block, __fancyvrb.number,
     __pygments, default, __fancyvrb,
  }
%    \end{MacroCode}
% Create a |\CDR_tags_if_display:...| set of conditionals.
%    \begin{MacroCode}
  \bool_if:nTF { \CDR_tag_if_truthy_p:c { show~tags }
      && ( ! \CDR_tag_if_truthy_p:c { only~top }
        || ! \CDR_clist_if_eq_p:NN \g_CDR_tags_clist \g_CDR_last_tags_clist
      )
  } {
    \prg_set_conditional:Nnn \CDR_tags_if_display: { p, T, F, TF } {
      \prg_return_true:
    }
  } {
    \prg_set_conditional:Nnn \CDR_tags_if_display: { p, T, F, TF } {
      \prg_return_false:
    }
  }
%    \end{MacroCode}
% For each \metatt{tag name}, create an \pkg{l3int} variable and intilize it to 1.
%    \begin{MacroCode}
  \clist_map_inline:Nn \g_CDR_tags_clist {
    \CDR_int_if_exist:cF { ##1 } {
      \CDR_int_new:cn { ##1 } { 1 }
    }
  }
}
%    \end{MacroCode}
%
% \begin{function}{\CDR_block_setup_pygments:}
% Utility to setup the |\l_CDR_pyg_bool| variable.
% \end{function}
%    \begin{MacroCode}
\cs_new:Npn \CDR_block_setup_pyg: {
  \bool_set_false:N \l_CDR_pyg_bool
  \CDR_tag_if_exist_here:ccTF { __local } { pygments } {
    \bool_set:Nn \l_CDR_pyg_bool {
      \CDR_tag_if_truthy_p:c { pygments }
    }  
  } {
    \clist_if_empty:NTF \g_CDR_tags_clist {
      \bool_set:Nn \l_CDR_pyg_bool {
        \CDR_tag_if_truthy_p:c { pygments }
      }
    } {
      \clist_map_inline:Nn \g_CDR_tags_clist {
        \CDR_tag_if_truthy:ccT { ##1 } { pygments } {
          \clist_map_break:n {
            \bool_set_true:N \l_CDR_pyg_bool
          }
        }
      }
    }
  }
}
%    \end{MacroCode}
%
% \begin{function}{\FVB@CDRBlock}
% \pkg{fancyvrb} helper to begin the environment.
% \end{function}
%    \begin{MacroCode}
\def\FVB@CDRBlock {
  \@bsphack
  \group_begin:
  \prg_set_conditional:Nnn \CDR_if_block: { p, T, F, TF } {
    \prg_return_true:
  }  
  \CDR_tag_keys_set:nn { __block } { __initialize }
%    \end{MacroCode}
% Reading the options: we absorb the options available in |\FV@KeyValues|,
% first for % \pkg{l3keys} modules, then for |\fvset|.
%    \begin{MacroCode}
  \CDR_keys_inherit:Vnn \c_CDR_tag { __local } {
    __block, __pygments.block, default.block,
    __pygments, default,
  }
  \CDR_tag_keys_set_known:nVN { __local } \FV@KeyValues \l_CDR_kv_clist
  \CDR_tag_provide_from_kv:V \l_CDR_kv_clist
  \CDR_tag_keys_set_known:nVN { __local } \l_CDR_kv_clist \l_CDR_kv_clist
%    \end{MacroCode}
% By default, this code chunk will have the same list of tags
% as the last code block or last |\CDRExport| stored in |\g_CDR_tags_clist|.
% This can be overwritten with the |tags=...| user interface.
% At least one tag must be provided.
%    \begin{MacroCode}
  \CDR_block_setup_tags:
  \lua_now:n {
    CDR:hilight_block_setup('g_CDR_tags_clist')
  }
%    \end{MacroCode}
% |\l_CDR_pyg_bool| is |true| iff one of the tags needs \pkg{pygments} or there is no tag and |pygments=true| was given.
%    \begin{MacroCode}
  \CDR_block_setup_pyg:
%    \end{MacroCode}
% And we can setup the engine.
%    \begin{MacroCode}
  \CDR_tag_get:cN { engine } \l_CDR_engine_tl
  \CDR_if_code_ngn:VF \l_CDR_engine_tl {
    \PackageError
      { coder }
      { \l_CDR_engine_tl\space block~engine~unknown,~replaced~by~'default' }
      {See~\CDRBlockEngineNew~in~the~coder~manual}
    \tl_set:Nn \l_CDR_engine_tl { default }
  }
  \CDR_tag_get:cN { \l_CDR_engine_tl~engine~options } \l_CDR_opts_tl
  \CDR_tag_get:cN { engine~options } \l_CDR_tl
  \tl_if_empty:NF \l_CDR_tl {
    \tl_put_right:Nn \l_CDR_opts_tl {,}
    \tl_put_right:NV \l_CDR_opts_tl \l_CDR_tl
  }
  \exp_args:NnV
  \use:c { \CDR_block_ngn:V \l_CDR_engine_tl } \l_CDR_opts_tl
%    \end{MacroCode}
% Now we branch according whether \pkg{pygments} is used or not.
%    \begin{MacroCode}
  \bool_if:NTF \l_CDR_pyg_bool {
    \CDRBlock@Pyg
  } {
    \CDRBlock@FV
  }
  \FV@Scan
}
\def\FVE@CDRBlock {
  \bool_if:NTF \l_CDR_pyg_bool {
    \endCDRBlock@Pyg
  } {
    \endCDRBlock@FV
  }
  \use:c { end \CDR_block_ngn:V \l_CDR_engine_tl }
  \group_end:
  \@esphack
}
\DefineVerbatimEnvironment{CDRBlock}{CDRBlock}{}


\newenvironment{CDRBlock@Pyg}{
%    \end{MacroCode}
% Catch the keys related to numbering before they are forwarded to \pkg{fancyvrb}.
%    \begin{MacroCode}
  \CDR_keys_inherit:Vnn \c_CDR_tag { __local } {
    __fancyvrb.number
  }
  \CDR_tag_keys_set_known:nVN { __local } \l_CDR_kv_clist \l_CDR_kv_clist
  \CDR_keys_inherit:Vnn \c_CDR_tag { __local } {
    __fancyvrb, __fancyvrb.block
  }
  \exp_args:NnV
  \CDR_tag_keys_set:nn { __local } \l_CDR_kv_clist
  \exp_args:NNV
  \def \FV@KeyValues \l_CDR_kv_clist
%    \end{MacroCode}
% Get the list of tags and setup \CDRLua{} for recording or hilighting.
%    \begin{MacroCode}
  \CDR_tag_get:cN {lang} \l_CDR_tl
  \lua_now:n { CDR:hilight_set_var('lang') }
  \CDR_tag_get:cN {cache} \l_CDR_tl
  \lua_now:n { CDR:hilight_set_var('cache') }
  \CDR_tag_get:cN {debug} \l_CDR_tl
  \lua_now:n { CDR:hilight_set_var('debug') }
  \CDR_tag_get:cN {style} \l_CDR_tl
  \lua_now:n { CDR:hilight_set_var('style') }
  \CDR@StyleIfExist { \l_CDR_tl } { } {
    \lua_now:n { CDR:hilight_source(true, false) }
    \input { \l_CDR_pyg_sty_tl }
  }
  \CDR@StyleUseTag
  \clist_map_inline:nn { firstnumber, stepnumber } {
    \CDR_tag_get:cN {##1} \l_CDR_tl
    \lua_now:n { CDR:hilight_set_var('##1') }
  }
  \CDR_tag_get:cN {stepnumber} \l_CDR_tl
  \lua_now:n { CDR:hilight_set_var('stepnumber') }
  \CDR_tag_if_truthy:cTF {no~export} {
    \clist_map_inline:nn { i, ii, iii, iv } {
      \cs_set:cpn { FV@ListProcessLine@ ##1 } ####1 {
        \tl_set:Nn \l_CDR_tl { ####1 }
        \lua_now:n { CDR:record_line('l_CDR_tl') }
      }
    }
  } {
    \clist_map_inline:nn { i, ii, iii, iv } {
      \cs_set:cpn { FV@ListProcessLine@ ##1 } ####1 {
        \tl_set:Nn \l_CDR_tl { ####1 }
        \lua_now:n { CDR:record_line('l_CDR_tl') }
      }
    }
  }
  \def\FV@ProcessLine ##1 {
    \tl_set:Nn \l_CDR_tl { ##1 }
    \lua_now:n { CDR:record_line('l_CDR_tl') }
  }
} {
  \CDR_tag_get:c { format }
  \fvset{ commandchars=\\\{\} }
  \CDR@DefineSp
  \FV@VerbatimBegin
  \lua_now:n { CDR:hilight_source(false, true) }
  \makeatletter
  \input{ \l_CDR_pyg_tex_tl }
  \makeatother
  \FV@VerbatimEnd
}

\newenvironment{CDRBlock@FV}{
%    \end{MacroCode}
% \pkg{pygments} is not used, \pkg{fancyvrb} features.
% We record the key value options about numbering.
%    \begin{MacroCode}
  \exp_args:NnV
  \use:c { \CDR_block_ngn:V \l_CDR_engine_tl } \l_CDR_opts_tl
  \CDR_keys_inherit:Vnn \c_CDR_tag { __local } {
    __fancyvrb.number,
  }
  \CDR_tag_keys_set_known:nVN { __local } \l_CDR_kv_clist \l_CDR_clist
  \CDR_tag_inherit:cn { __local } {
    __fancyvrb.number,
    __block,
  }
  \CDR_tag_if_truthy:cF {no~export} {
    \clist_map_inline:nn { i, ii, iii, iv } {
      \cs_set:cpn { FV@ListProcessLine@ ##1 } ####1 {
        \tl_set:Nn \l_CDR_tl { ####1 }
        \lua_now:n { CDR:record_line('l_CDR_tl') }
        \use:c { CDR@ListProcessLine@ ##1 } { ####1 }
      }
    }
  }
%    \end{MacroCode}
% Prepare the counters. The |__| int starts with 0, which means that it is unused.
% If the |firstnumber| value is ``last'' then it is used to store the first number.
%    \begin{MacroCode}
  \CDR_int_set:cn { __ } { 0 }
  \CDR_tag_if_eq:cnF { numbers } { none } {
    \CDR_tag_if_eq:cnT { firstnumber } { last } {
      \clist_map_inline:Nn \g_CDR_tags_clist {
        \clist_map_break:n {
          \CDR_int_set:cc { __ } { ##1 }
          \clist_put_right:Nx \l_CDR_kv_clist {
            firstnumber = \CDR_int_use:c { ##1 }
          }
        }
      }
    }
  }
  \exp_args:NNV
  \def \FV@KeyValues \l_CDR_kv_clist
  \FV@VerbatimBegin
}{
  \FV@VerbatimEnd
  \CDR_tag_if_eq:cnF { numbers } { none } {
    \CDR_int_compare:cNnTF { __ } > 0 {
      \CDR_int_set:cn { __ } {
        \value{FancyVerbLine} - \CDR_int_use:c { __ } + 1
      }
      \clist_map_inline:Nn \g_CDR_tags_clist {
        \CDR_int_gadd:cc { ##1 } { __ }
      }
    } {
      \CDR_int_set:cn { __ } { \value{FancyVerbLine} + 1 }
      \clist_map_inline:Nn \g_CDR_tags_clist {
        \CDR_int_gset:cc { ##1 } { __ }
      }
    }
  }
}

\prg_new_conditional:Nnn \CDR_clist_if_eq:NN { p, T, F, TF } {
  \tl_if_eq:NNTF #1 #2 {
    \prg_return_true:
  } {
    \prg_return_false:
  }
}

\ExplSyntaxOff
\makeatother

\subsubsection{\textsf{fancyvrb} linear}

\typeout{IN PROGREEEEEESSSSSSSS}
\CDRSet{pygments=false}

\begin{Verbatim} [
  numbers=none,
]
A
\end{Verbatim}

\begin{CDRBlock} [
  tags=none,
  numbers=none,
]
A
\end{CDRBlock}
\begin{CDRBlock} [
  tags=none,
  numbers=left,
  firstnumber=last,
]
A
B
C
\end{CDRBlock}
\begin{CDRBlock} [
  tags=none,
  numbers=left,
  firstnumber=last,
]
A
B
C
\end{CDRBlock}
\begin{CDRBlock} [
  tags=none,
  numbers=left,
  firstnumber=last,
]
A
B
C
\end{CDRBlock}

\subsubsection{\textsf{fancyvrb} multi tags}

\CDRSet{pygments=false}

\begin{CDRBlock} [
  tags={A,B,C},
  numbers=left,
  firstnumber=last,
]
A=*1,B=1,C=1
\end{CDRBlock}
\begin{CDRBlock} [
  tags={B,C},
  numbers=left,
  firstnumber=last,
]
A=2,B=*2,C=2
A=2,B=*3,C=3
\end{CDRBlock}

\begin{CDRBlock} [
  tags=C,
  numbers=left,
  firstnumber=last,
]
A=2,B=4,C=*4
A=2,B=4,C=*5
\end{CDRBlock}

\begin{CDRBlock} [
  tags={C, B},
  numbers=left,
  firstnumber=last,
]
A=2,B=4,C=*6
A=2,B=5,C=*7
A=2,B=6,C=*8
\end{CDRBlock}

\begin{CDRBlock} [
  tags={B, A},
  numbers=left,
  firstnumber=last,
]
A=2,B=*7,C=9
A=3,B=*8,C=9
A=4,B=*9,C=9
\end{CDRBlock}
\begin{CDRBlock} [
  tags={A,C},
  numbers=left,
  firstnumber=last,
]
A=*5,B=10,C=9
A=*6,B=10,C=10
A=*7,B=10,C=11
\end{CDRBlock}

\ExplSyntaxOn
\CDR_int_compare:cNnF { A } = 8 { FAILED \\ }
\CDR_int_compare:cNnF { B } = {10} { FAILED \\ }
\CDR_int_compare:cNnF { C } = {12} { FAILED \\ }
\ExplSyntaxOff

\subsection{\textsf{fancyvrb} properties}

\begin{CDRBlock} [
  tags=none,
  numbers=left,
  firstnumber=last,
]
A
B
C
\end{CDRBlock}

\newpage

\subsection{Display tags}

\makeatletter
\ExplSyntaxOn

\cs_undefine:N \CDR@Line
\cs_undefine:N \CDR_line:nnn
\cs_undefine:N \CDR_line:nn
\cs_undefine:N \CDR@NumberFormat
\cs_undefine:N \CDR@TagsFormat
\cs_undefine:N \CDR@NumberSep
\cs_undefine:N \CDR_info:n

\cs_new:Npn \CDR@Line {
  \peek_meaning_ignore_spaces:NTF [ { \CDR_line:nnn } { \CDR_line:nn }
}
%    \end{MacroCode}
% \begin{function}{\CDR_line:nnn}
% \begin{syntax}
% \cs{CDR_line:nnn} \Arg{CDR@Line kv list} \Arg{ine number} \Arg{line commands}
% \end{syntax}
% This is the very first command called when typesetting.
% Some setup are made for line numbering, in particular the |\CDR_int_if_visible:n...|
% family is set here.
% \end{function}
%    \begin{MacroCode}
\cs_new:Npn \CDR_line:nnn [ #1 ] {
  \keys_set:nn { CDR@Line } { #1 }
  \int_compare:nNnTF { \CDR_int:c { __step } } = 1 {
    \tl_set:Nn \l_CDR_tl { {
      \prg_return_true:
    } }
  } {
    \tl_set:Nn \l_CDR_tl { = { ##1 } {
        \prg_return_true:
      } {
        \prg_return_false:
      }
    }
    \int_compare:nNnTF { \CDR_int:c { __step } / 5 * 5 } = { \CDR_int:c { __step } } {
      \tl_put_left:Nn \l_CDR_tl { {
        ( ##1 - \CDR_int:c { __start } ) / \CDR_int:c { __step } / \CDR_int:c { __step }
      } }  
    } {
      \tl_put_left:Nn \l_CDR_tl { {
        ( ##1 ) / \CDR_int:c { __step } / \CDR_int:c { __step }
      } }
    }
    \tl_put_left:Nn \l_CDR_tl { \int_compare:nNnTF }
%    \end{MacroCode}
% \begin{function}[EXP,pTF]{\CDR_int_if_visible:n}
% \begin{syntax}
% \cs{CDR_int_if_visible:nTF} \Arg{line number} \Arg{true code} \Arg{false code}
% \end{syntax}
% Execute \metatt{true code} if the \metatt{linenumber} is visible, \metatt{false code} otherwise.
% The \metatt{linenumber} visibility depends on the value relative to first number and the step.
% This is relavant only when line numbering is enabled.
% Some setup are made for line numbering, in particular the |\CDR_int_if_visible:n...|.
% family is set here.
% \end{function}
%    \begin{MacroCode}
  }
  \exp_args:NNnV
  \prg_set_conditional:Nnn \CDR_int_if_visible:n { p, T, F, TF } \l_CDR_tl
  \CDR_line:nn
}
\cs_new:Npn \CDR_line:nn #1 #2 {
  \typeout{CDR@Line:\the\leftmargin/#1/}
  \hbox to \hsize {
    \kern \leftmargin
    \CDR_info:n { #1 }
    \hbox to \linewidth {
      \FV@LeftListFrame
      #2
      \hss
      \FV@RightListFrame
    }
  }
}

\cs_new:Npn \CDR@NumberFormat {
  \CDR_tag_get:c { numbers~format }
}
\cs_new:Npn \CDR@TagsFormat {
  \CDR_tag_get:c { tags~format }
}
\cs_new:Npn \CDR@NumberSep {
  \hspace{ \CDR_tag_get:c { numbersep } }
}

\cs_new:Npn \CDR_info_body:n { \CDR_info_body_a:n }
%\cs_set:Npn \CDR_info_body:n { \use_none:n }

\CDR_int_new:cn { __start } { 0 }
\CDR_int_new:cn { __step } { 0 }
%    \end{MacroCode}
% \begin{function}{ \CDR_number_display:n }
% First line.
% \end{function}
%    \begin{MacroCode}
\cs_set:Npn \CDR_number_display_a:n #1 {
  \int_compare:nNnTF { (#1/\FV@StepNumber)*\FV@StepNumber} = {#1} {
  } {
  }
  \@tempcnta=\FV@CodeLineNo
  \@tempcntb=\FV@CodeLineNo
  \divide\@tempcntb\FV@StepNumber
  \multiply\@tempcntb\FV@StepNumber
  \ifnum\@tempcnta=\@tempcntb\relax\else\fi
}
%    \end{MacroCode}
% \begin{function}{\CDR_info_body_a:n}
% First line.
% \end{function}
%    \begin{MacroCode}
\cs_new:Npn \CDR_info_body_a:n #1 {
  \CDR_int_set:cn { __ } { 0 }
  \CDR_tag_if_eq:cnF { numbers } { none } {
    \CDR_tag_if_eq:cnTF { firstnumber } { last } {
      \clist_map_inline:Nn \g_CDR_tags_clist {
        \clist_map_break:n {
          \CDR_int_set:cc { __start } { ##1 }
        }
      }
    } {
      \CDR_tag_if_eq:cnTF { firstnumber } { auto } {
        \CDR_int_set:cn { __start } { 1 }
      } {
        \exp_args:Nnf
        \CDR_int_set:cn { __start } { \CDR_tag_get:c { firstnumber } }
      }
    }
    \typeout{STEP: \CDR_tag_get:c { stepnumber } }
    \exp_args:Nnx
    \CDR_int_set:cn { __step } { \CDR_tag_get:c { stepnumber } }
  }
  \CDR_tags_if_display:TF {
    \smash{
      \parbox[b]{\marginparwidth}{
        \raggedleft
        { \CDR@TagsFormat \g_CDR_tags_clist :}
        \CDR_tag_if_eq:cnTF { numbers } { none } {
          \cs_set:Npn \CDR_info_body:n { \use_none:n }
        } {
          \CDR_int_compare:cNnT { __last } = 1 {
            {}~#1
          }
          \cs_set:Npn \CDR_info_body:n { \CDR_info_body_ba:n }
        }
      }
    }
    \clist_gset_eq:NN \g_CDR_last_tags_clist \g_CDR_tags_clist
  } {
    \CDR_tag_if_eq:cnTF { numbers } { none } {
      \cs_set:Npn \CDR_info_body:n { \use_none:n }
    } {
      #1
      \cs_set:Npn \CDR_info_body:n { \CDR_info_body_bb:n }
    }
  }
}
%    \end{MacroCode}
% \begin{function}{\CDR_info_body_a:n}
% First line.
% \end{function}
%    \begin{MacroCode}


%'''    if len(lines):
%      more('First', lines.pop(0))
%      if len(lines):
%        more('Second', lines.pop(0))
%        if stepnumber < 2:
%          def template():
%            return 'Black'
%        elif stepnumber % 5 == 0:
%          def template():
%            return 'Black' if number %\
%              stepnumber == 0 else 'White'
%        else:
%          def template():
%            return 'Black' if (number - firstnumber) %\
%              stepnumber == 0 else 'White'
%'''         

%    \end{MacroCode}
% Display the line number unless the step is >0 and the next line number is displayed.
%    \begin{MacroCode}
\cs_new:Npn \CDR_info_body_ba:n #1 {


}

\cs_new:Npn \CDR_info_body_bb:n #1 {

}

\cs_new:Npn \CDR_info:n #1 {
  \hbox_overlap_left:n {
    \cs_set:Npn \baselinestretch { 1 }
    { \CDR@NumberFormat
      \CDR_info_body:n { #1 }
    }
    \CDR@NumberSep
  }
}

\ExplSyntaxOff
\makeatother

\typeout{DEBUGGGGG1}
\begin{CDRBlock}[
  tags={we are the champions,  whatever it isp},
  fontfamily=menlo,
  fontsize=\large,
  pygments=true,
  lang=lua,
  numbers=left,
  debug=true,
  show tags
]
function foo(arg) return arg ** arg end -- tags expected
\end{CDRBlock}
\typeout{DEBUGGGGG2}
\begin{CDRBlock}[
%  tags={we are the champions,  whatever it isp},
  fontfamily=menlo,
  fontsize=\large,
  pygments=true,
  lang=lua,
  numbers=left,
  frame=lines,
  debug=true,
  frame=single,
  framerule=1mm,
  framesep=3mm,
  rulecolor=\color{red},
  fillcolor=\color{yellow},
  showspaces,
  baselinestretch=1.75
]
function foo(arg) -- no tags expected
  return arg ** arg
end
\end{CDRBlock}
\typeout{DEBUGGGGG3}
\begin{CDRBlock}[
%  tags={we are the champions,  whatever it isp},
  fontfamily=menlo,
  fontsize=\large,
  pygments=true,
  lang=lua,
  numbers=left,
  debug=true,
  show tags,
  only top=false
]
function foo(arg) return arg ** arg end -- tags expected
\end{CDRBlock}
\typeout{DEBUGGGGG4}
\begin{CDRBlock}[
  tags={whatever it isp,we are the champions},
  fontfamily=menlo,
  fontsize=\large,
  pygments=true,
  lang=lua,
  numbers=left,
  debug=true,
  show tags,
  only top
]
function foo(arg) return arg ** arg end -- tags expected
\end{CDRBlock}
\typeout{DEBUGGGGG5}
\begin{CDRBlock}[
%  tags={we are the champions,  whatever it isp},
  pygments,
  lang=lua,
  numbers=none,
  debug,
  showspaces,
  show tags=false,
  only top=false,
]
function foo(arg) -- no tags expected
  -- second line
  -- third line
  -- fourth line
  -- fifth line
  return arg ** arg
end
\end{CDRBlock}

\egroup

\CDRSet{pygments=false}
